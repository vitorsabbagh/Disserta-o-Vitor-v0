\chapter{Conclusions}

    The results of this study highlight the potential of multi-agent architectures based on LLMs in the O\&G sector, particularly in the domain of well engineering. The ability to process and respond to complex queries paves the way for a significant digital transformation in the area.

    Our comparative analysis of single-agent and multi-agent architectures, using GPT-3.5-turbo and GPT-4, reveals a detailed landscape of trade-offs between performance and economic efficiency. Multi-agent systems demonstrate 28\% greater factuality in question-and-answer (Q\&A) tasks, especially with GPT-4, compared to single-agent systems. However, they incur LLM costs that are, on average, 3.7 times higher due to the complexities of inter-agent communication. In contrast, single-agent systems excel in Text-to-SQL tasks, showing 15\% better performance than multi-agent setups. This cost-benefit dynamic requires careful consideration when implementing RAG in real-world scenarios, where accuracy and financial constraints must be balanced.

    We highlight several challenges encountered during our experiments, including issues with contextualization, the need for more refined information filtering, and the persistence of hallucinations. These challenges underscore the need for continuous research in areas such as domain-specific specialized models, advanced semantic search techniques, and hybrid architectures that combine the strengths of single-agent and multi-agent systems.

    The practical implications of this study extend beyond the O\&G sector. The insights gained here are applicable to any knowledge-intensive domain that deals with large volumes of technical data. By focusing on enhancing retrieval mechanisms, developing domain-specific LLMs, and optimizing interactions between agents and tools, we pave the way for more effective, reliable, and cost-efficient RAG solutions across various sectors.

    The main points of the study are as follows: multi-agent systems offer superior factuality in Q\&A tasks, albeit at a significantly higher cost. Single-agent architectures, on the other hand, excel at Text-to-SQL tasks. Despite the advantages, several challenges persist, including issues with contextualization, filtering, hallucination, and domain-specific vocabulary.

    Future research should focus on developing specialized models, advancing retrieval techniques, and exploring hybrid architectures. The lessons learned from this study have broader implications and can extend to other complex technical domains. By addressing the limitations identified in this study and embracing emerging trends in multi-agent systems and RAG technology, we can unlock their full potential, revolutionizing decision-making, knowledge management, and operational efficiency in complex industries worldwide.



    [CONCLUSIONS FROM EXPERIMENT 2 ARE YET TO BE WRITTEN HERE]

    [INCLUIR: COMENTAR QUE OS PRECOS DE LLM ESTAO CAINDO MTO, MODELOS PEQUENOS COM EXCELENTE DESEMPENHO, O QUE TORNA A ANALISE FINANCEIRA DO 1o CICLO OBSOLETA]