\begin{figure}[htbp]
    \centering
    \begin{tikzpicture}[
        node distance=2.5cm and 1.5cm,
        every node/.style={
            shape=rectangle,
            rounded corners,
            draw,
            align=center,
            minimum height=1.2cm,
            minimum width=2.5cm,
            thick
        },
        arrow/.style={
            ->,
            >=stealth,
            thick
        },
        io/.style={
            shape=rectangle,
            draw=none,
            align=center
        }
    ]
        % Nodes
        \node[io] (user) {User};
        \node (query) [below=0.5cm of user] {User Query};
        \node (retriever) [right=of query] {Retriever};
        \node (knowledge) [above=of retriever] {Knowledge Base};
        \node (generator) [right=of retriever, text width=3cm] {Generator (LLM)};
        \node (response) [right=of generator] {Generated Response};
        \node[io] (output_user) [below=0.5cm of response] {User};

        % Arrows
        \draw[arrow] (user) -- (query);
        \draw[arrow] (query) -- (retriever);
        \draw[arrow] (knowledge) -- (retriever) node[midway, right, text width=2cm] {Relevant Documents};
        
        % Augmented Prompt
        \draw[arrow] (retriever.east) -- ++(0.75,0) -- ++(0,1.25) -| (generator.north) node[midway, above] {Context};
        \draw[arrow] (query.east) -- ++(0.75,0) -- ++(0,-1.25) -| (generator.south) node[midway, below] {Query};
        
        \draw[arrow] (generator) -- (response);
        \draw[arrow] (response) -- (output_user);

    \end{tikzpicture}
    \caption{Diagram of the Retrieval-Augmented Generation (RAG) process.}
    \label{fig:rag_diagram}
\end{figure}
