

\chapter{Conclusões 1}

    Os resultados deste estudo destacam o potencial das arquiteturas multiagente baseadas em LLMs no setor de O\&G, especialmente no domínio da engenharia de poços. A capacidade de processar e responder a consultas complexas abre caminho para uma transformação digital significativa na área.
    
    Nossa análise comparativa de arquiteturas de agente único e multiagente, utilizando GPT-3.5-turbo e GPT-4, revela um panorama detalhado de trade-offs entre desempenho e eficiência econômica. Os sistemas multiagente demonstram uma veracidade 28\% maior em tarefas de perguntas e respostas (Q\&A), especialmente com GPT-4, em comparação com sistemas de agente único. No entanto, eles incorrem em custos de LLM que são, em média, 3,7 vezes maiores devido às complexidades da comunicação entre agentes. Em contraste, os sistemas de agente único se destacam em tarefas de Text-to-SQL, apresentando um desempenho 15\% melhor do que as configurações multiagente. Essa dinâmica de custo-benefício exige uma consideração cuidadosa ao implementar RAG em cenários do mundo real, onde precisão e restrições financeiras devem ser equilibradas.
    
    Destacamos vários desafios encontrados durante nossos experimentos, incluindo questões de contextualização, necessidade de filtragem de informações mais refinada e a persistência de alucinações. Esses desafios sublinham a necessidade de pesquisas contínuas em áreas como modelos especializados em domínios específicos, técnicas avançadas de busca semântica e arquiteturas híbridas que combinem as forças dos sistemas de agente único e multiagente.
    
    As implicações práticas deste estudo vão além do setor de O\&G. Os insights alcançados aqui são aplicáveis a qualquer domínio intensivo em conhecimento que lide com grandes volumes de dados técnicos. Ao focar em aprimorar os mecanismos de recuperação, desenvolver LLMs específicos de domínio e otimizar as interações entre agentes e ferramentas, pavimentamos o caminho para soluções RAG mais eficazes, confiáveis e econômicas em diversos setores.
    
    Os principais pontos do estudo são os seguintes: sistemas multiagente oferecem superior veracidade em tarefas de Q\&A, embora a um custo significativamente maior. Arquiteturas de agente único, por outro lado, se destacam em tarefas de Text-to-SQL. Apesar das vantagens, persistem vários desafios, incluindo questões de contextualização, filtragem, alucinação e vocabulário específico de domínio.
    
    Pesquisas futuras devem focar no desenvolvimento de modelos especializados, no avanço das técnicas de recuperação e na exploração de arquiteturas híbridas. As lições aprendidas deste estudo têm implicações mais amplas e podem se estender a outros domínios técnicos complexos. Ao abordar as limitações identificadas neste estudo e abraçar as tendências emergentes em sistemas multiagente e tecnologia RAG, podemos desbloquear seu potencial total, revolucionando a tomada de decisões, a gestão do conhecimento e a eficiência operacional em indústrias complexas em todo o mundo.
    
    