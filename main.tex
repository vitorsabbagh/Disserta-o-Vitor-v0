%% This is a monograph which uses `coppe' document
%% class and `coppe-unsrt' BibTeX style.
%% 
%% \CheckSum{1648}
%% \CharacterTable
%%  {Upper-case    \A\B\C\D\E\F\G\H\I\J\K\L\M\N\O\P\Q\R\S\T\U\V\W\X\Y\Z
%%   Lower-case    \a\b\c\d\e\f\g\h\i\j\k\l\m\n\o\p\q\r\s\t\u\v\w\x\y\z
%%   Digits        \0\1\2\3\4\5\6\7\8\9
%%   Exclamation   \!     Double quote  \"     Hash (number) \#
%%   Dollar        \$     Percent       \%     Ampersand     \&
%%   Acute accent  \'     Left paren    \(     Right paren   \)
%%   Asterisk      \*     Plus          \+     Comma         \,
%%   Minus         \-     Point         \.     Solidus       \/
%%   Colon         \:     Semicolon     \;     Less than     \<
%%   Equals        \=     Greater than  \>     Question mark \?
%%   Commercial at \@     Left bracket  \[     Backslash     \\
%%   Right bracket \]     Circumflex    \^     Underscore    \_
%%   Grave accent  \`     Left brace    \{     Vertical bar  \|
%%   Right brace   \}     Tilde         \~}
%%
\documentclass[msc,english]{coppe}

\usepackage{booktabs}% tabelas mais bonitas
\usepackage{rotating}% rodando coisas, como tabelas
\usepackage{longtable} % tabelas longas
\usepackage[most]{tcolorbox} % caixas de texto
\usepackage{amsmath,amssymb}
\usepackage{hyperref}

\usepackage{multirow}
\usepackage{changepage} 

\usepackage{adjustbox} 

\usepackage{xcolor, colortbl} % For more complex coloring of cells
\usepackage{hhline} % For double lines

\usepackage{algorithm}
% \usepackage{algorithmic}
\usepackage{algpseudocode}

\usepackage{lmodern}
\usepackage[T1]{fontenc}

\usepackage{silence}
% \WarningFilter{latex}{Overfull}
\WarningFilter{latex}{Underfull}
\WarningFilter{latex}{empty journal}

% \usepackage{float}
\usepackage{pdflscape} % in preamble

\usepackage{listings}

\definecolor{codegray}{rgb}{0.5,0.5,0.5}
\definecolor{keywordgreen}{rgb}{0.1,0.4,0.2}

\lstdefinestyle{mystyle}{ 
    commentstyle=\color{codegray},
    keywordstyle=\color{keywordgreen}\bfseries,
    numberstyle=\color{codegray},
    stringstyle=\color{purple},
    basicstyle=\ttfamily\small,
    breakatwhitespace=false,         
    breaklines=true,                 
    captionpos=b,                    
    keepspaces=true,                 
    numbers=left,                    
    numbersep=10pt,                  
    showspaces=false,                
    showstringspaces=false,
    showtabs=false,                  
    tabsize=2
}

% \usepackage[a4paper,margin=1.5cm]{geometry}


\makelosymbols
\makeloabbreviations

\begin{document}


\title{Comparative Analysis of Single and Multi-Agent Large Language Model Architectures for Domain-Specific Tasks in Well Construction}
\foreigntitle{Comparative Analysis of Single and Multi-Agent Large Language Model Architectures for Domain-Specific Tasks in Well Construction}
\author{Vitor}{Brandão Sabbagh}
\advisor{Prof.}{Geraldo}{Bonorino Xexéo}{D.Sc.}

\examiner{Prof.}{Nome do Primeiro Examinador Sobrenome}{D.Sc.}
\examiner{Prof.}{Nome do Segundo Examinador Sobrenome}{Ph.D.}
\examiner{Prof.}{Nome do Terceiro Examinador Sobrenome}{D.Sc.}
\examiner{Prof.}{Nome do Quarto Examinador Sobrenome}{Ph.D.}
\examiner{Prof.}{Nome do Quinto Examinador Sobrenome}{Ph.D.}
\department{PESC}
\date{07}{2025}

\keyword{Large Language Models}
\keyword{Agents}
\keyword{Oil Well Construction}

\maketitle

\frontmatter
\dedication{To Carolina, my life partner.}

\chapter*{Acknowledgements}

I would like to thank everyone. [family, friends, etc]

I extend my gratitude to the well construction engineering experts, Marcelo Grimberg, Rafael Peralta, and Lorenzo Simonassi, whose expertise and dedication significantly contributed to this research.


\begin{abstract}

Apresenta-se, nesta tese, a aplicação de grandes modelos de linguagem (LLM) no setor de petróleo e gás, especificamente em tarefas de construção e manutenção de poços. O estudo avalia o desempenho de uma arquitetura baseada em LLM de agente único e de múltiplos agentes no processamento de diferentes tarefas, oferecendo uma perspectiva comparativa sobre sua precisão e as implicações de custo de sua implementação. Os resultados indicam que sistemas multiagentes oferecem desempenho melhorado em tarefas de perguntas e respostas, com uma medida de veracidade 28\% maior do que os sistemas de agente único, mas a um custo financeiro mais alto. Especificamente, a arquitetura multiagente incorre em custos que são, em média, 3,7 vezes maiores do que os da configuração de agente único, devido ao aumento do número de tokens processados. Por outro lado, os sistemas de agente único se destacam em tarefas de texto para SQL (Linguagem de Consulta Estruturada), especialmente ao usar o Transformador Pré-Treinado Generativo 4 (GPT-4), alcançando uma pontuação 15\% maior em comparação com as configurações multiagentes, sugerindo que arquiteturas mais simples podem, às vezes, superar a complexidade. A novidade deste trabalho reside em seu exame original dos desafios específicos apresentados pelos dados complexos, técnicos e não estruturados inerentes às operações de construção de poços, contribuindo para o planejamento estratégico da adoção de aplicações de IA generativa, fornecendo uma base para otimizar soluções contra parâmetros econômicos e tecnológicos.

\end{abstract}

\begin{foreignabstract}


This article explores the application of large language models (LLM) in the oil and gas  sector, specifically within well construction and maintenance tasks. The study evaluates the performances of a single-agent and a multi-agent LLM-based architecture in processing different tasks, offering a comparative perspective on their accuracy and the cost implications of their implementation. The results indicate that multi-agent systems offer improved performance in question and answer tasks, with a truthfulness measure 28\% higher than single-agent systems, but at a higher financial cost. Specifically, the multi-agent architecture incurs costs that are, on average, 3.7 times higher than those of the single-agent setup due to the increased number of tokens processed. Conversely, single-agent systems excel in text-to-SQL (Structured Query Language) tasks, particularly when using Generative Pre-Trained Transformer 4 (GPT-4), achieving a 15\% higher score compared to multi-agent configurations, suggesting that simpler architectures can sometimes outpace complexity. The novelty of this work lies in its original examination of the specific challenges presented by the complex, technical, unstructured data inherent in well construction operations, contributing to strategic planning for adopting generative AI applications, providing a basis for optimizing solutions against economic and technological parameters. 

\end{foreignabstract}

\tableofcontents
\listoffigures
\listoftables
\printlosymbols
\printloabbreviations

\mainmatter



\chapter{Introduction}


% \section{Contextualization}


    % [IA-GEN NA INDUSTRIA] 
    In the dynamic changing of the oil and gas (O\&G) industry, digital transformation has emerged as a key element to achieve operational efficiency, sustainability, and competitiveness. 
    At the forefront of this transformation are Large Language Models (LLMs), which have the potential to process unstructured queries, map out alternatives, and advise users on possible actions \cite{Kar2023}. 
    We also note the advantage of increased engagement, cooperation, accessibility, and ultimately profitability. 
    These models redefine paradigms in knowledge management and information retrieval and impact a variety of other areas \cite{Eckroth2023}, making it crucial to adopt these technologies to remain competitive.    
    
    % [ESTUDO AUMENTO PRODUTIVIDADE] 
    A study conducted by \cite{Dellacqua2023}, in collaboration with the Boston Consulting Group demonstrates that in knowledge-intensive tasks, consultants equipped with access to LLMs like GPT-4 not only completed tasks more efficiently (25.1\% more quickly on average) but also with substantially higher quality, achieving results more than 40\% better compared to those without AI assistance \cite{Dellacqua2023}. Increase in productivity of knowledge workers was 12\% on average.    
    A major oil company spent in 2023 \$2.8B with employee compensation. A potential increase of 12\% in knowledge workers productivity, given they represent 60\% of all employee, could represent \$204M annual savings in this scenario.     
    % [AUMENTO DO PIB DEVIDO A GEN AI] 
    Broader economic indicators predict significant transformations due to generative AI (Gen-AI) across various industries.
    A report from Goldman Sachs \cite{Hatzius2023} highlights that Gen-AI is poised to increase global GDP by nearly 7\%, increasing productivity growth by 1.5 percentage points over the next decade. 
    This economic uplift is expected due to AI's ability to automate complex workflows and create new business opportunities, significantly impacting employment and productivity sectors worldwide.
            
    % [PROBLEMA DE DADOS NA INDUSTRIA EM GERAL] 
    Expanding on the broader discussion on data utilization within organizations, an important issue is the challenge of extracting relevant information from extensive databases \cite{Singh2023}. 
    Initially, the challenge of knowing, finding, and accessing data poses a significant obstacle to decision-making processes. Collaborators at O\&G companies often face the intensive task of manually searching large data repositories to find useful information.
    
    % [PROBLEMA DE DADOS NO O\&G] 
    Focusing specifically on the activities of drilling and completion of offshore and onshore wells, a major challenge lies in the inherently complex and technical nature of the data involved, which can be from various types: operations, projects, technologies, supply chains, and others. 
    Inefficiency in leveraging large volumes of unstructured data worsens these challenges, as observed by \cite{Singh2023}. A significant amount of the data generated and collected in this sector is unstructured, ranging from text reports and emails to images and videos of exploration and production activities. 
    Examples include hundreds of daily operational reports from drilling rigs, well execution projects, nonproductive time (NPT) reports, and operational lessons learned documents, as illustrated in Figure \ref{fig:report_example}. 
    As a result, valuable information can remain untapped, and the potential to find insights, informed decision-making and innovation is significantly compromised.
    \cite{Singh2023} showcases the capabilities and potential of Generative AI-enabled chatbots for the O\&G sector, particularly in enhancing drilling and production analytics to achieve better business outcomes. The author concludes that companies that adopt these technologies in the coming years will see clear advantages.     
    
    \begin{figure}[t]
        \centering
        \includegraphics[width=1\textwidth]{images/report_example.png}
        \caption{Sample of drilling \& completion learned lesson partial document. (translated from Portuguese)}
        \label{fig:report_example}
    \end{figure}           
    
    However, the deployment of such technologies presents limitations and introduces challenges, including biased data, hallucinations, lack of explainability, and logical reasoning errors, among others \cite{Hadi2023}, which require a balanced approach to harness their potential in a responsible manner.    
    Although previous research has focused mainly on the broader applications of AI in industry, the novelty of our research lies in its original examination of the specific challenges and solutions presented by the complex, technical and unstructured data inherent in O\&G operations. By comparing single- and multi-agent systems, this study fills a knowledge gap, providing empirical insights into the effectiveness of different Gen-AI architectures in a domain where such studies are scarce. 
    
    The adoption of these technologies by a major oil company underscores their potential to revolutionize data analysis and management, presenting an opportunity for deeper exploration and application.


\section{Objectives}

    This research directly addresses challenges faced by major oil companies. By investigating the comparative advantages and limitations of various Gen-AI architectures, including single and multi-agent systems, for Q\&A and Text-to-SQL tasks, this study aims to identify the most efficient and cost-effective solutions.
    The specific objectives of this research are to assess the suitability and effectiveness of multi-agent systems based on LLMs for complex, domain-specific tasks in well engineering, aiming to streamline information access and decision-making. 
    The study will compare single-agent and multi-agent AI systems in terms of their ability to address well engineering queries. Finally, it will map the potential obstacles and limitations associated with deploying Gen-AI applications.
            
    The insights gained from this research will directly contribute to O\&G companies strategic goals by improving access to well engineering information and automated data analysis tasks. 
    A comprehensive understanding of the challenges and limitations associated with Gen-AI will enable informed decisions about its adoption, maximizing the return on investment. 

    To achieve these objectives, this research was conducted through two distinct experimental phases. The first, carried out in 2024, focused on a foundational comparison between single and multi-agent architectures, revealing key insights into their performance, cost, and limitations such as hallucination and context interpretation. The rapid evolution of generative AI frameworks and models prompted a second, more advanced experiment in 2025. This second phase built upon the initial findings, also employing non-agentic workflows as baseline and a more rigorous, quantitative evaluation methodology to address the challenges identified in the first experiment and automated evaluation based on the concept commonly reffered to as "LLM-as-judge" (\cite{gu2025surveyllmasajudge}).

\section{Business Scope Delimitation}

    To contextualize the scope of this study, it is necessary to understand the life cycle of an oil field, which begins with Exploration and progresses to the Development of Production, followed by effective production, and culminates in decommissioning \cite{Badiru2016}. Gen-AI has the potential to impact each of these phases, but the focus of this work lies in the operations of the development and maintenance stages.
            
    Well construction is a highly specialized activity involving drilling and completion of wells for hydrocarbon extraction \cite{Thomas2004}. In this context, Gen-AI can be applied in various ways. For example, a chatbot could manage knowledge by answering queries about operations and well projects by retrieving information from the organization's databases. Additionally, LLM-based agents could be used in executive project review to ensure that drilling or completion operations comply with the organization's standards and adhere to best operational practices. Moreover, Gen-AI could perform inference in unstructured databases to extract specific information from text reports and obtain structured data. 
    
    This business scope emphasizes the importance of Gen-AI in the construction and maintenance of wells.

\section{Thesis Structure}

    ****SERÁ FEITO POR ÚLTIMO****


% 
\chapter{Introdução 2 [CANCELADO]} 


    \section{Contextualização 2 [ok]}

        % The oil and gas industry is currently navigating a complex landscape characterized by significant challenges in well construction and maintenance. These challenges include managing vast amounts of data, ensuring operational efficiency, and maintaining safety standards. The integration of advanced technologies such as Large Language Models (LLMs) offers promising solutions to optimize processes and enhance decision-making in this sector. LLMs, with their ability to process and analyze large datasets, can significantly improve the efficiency and accuracy of well construction and maintenance operations. This section will explore the current challenges in the oil and gas industry, particularly in well construction, and introduce the potential of LLMs as transformative tools in this domain.
        
        ****USAR O ANTERIOR****

        
        \subsection{Challenges in Well Construction and Maintenance}
        
            % Data Management: 
            The oil and gas industry generates substantial volumes of data from various sources, including sensor data, reports, and historical records. However, much of this data remains underutilized due to difficulties in retrieval and analysis, leading to inefficiencies in well construction and maintenance operations \cite{Michael_Yi_2024}; \cite{Myriam_Amour_2024}.
            
            % Operational Efficiency: 
            The need for real-time decision-making in well construction is critical. Traditional methods often fall short in providing timely insights, which can lead to increased downtime and higher operational costs \cite{E_Ferrigno_2024}.
            
            % Safety and Environmental Concerns: 
            Ensuring safety and minimizing environmental impact are paramount in well construction. The complexity of operations necessitates robust data governance and standardization to maintain safety standards and comply with regulations \cite{Syatria_Kumala_Putra_2024}.
            Potential of LLMs in Optimizing Processes
            
        \subsection{Potential of LLMs in Optimizing Well Construction}
               
            % Enhanced Data Retrieval and Analysis: 
            LLMs can significantly improve the retrieval and analysis of well construction data by providing quick access to relevant information and facilitating complex data queries. This capability reduces the time required for data processing and enhances decision-making efficiency \cite{Michael_Yi_2024}; \cite{Myriam_Amour_2024}.
            
            % Real-Time Decision Support: 
            By integrating LLMs into drilling control rooms, companies can achieve real-time classification and analysis of data, which is crucial for responding to critical events during drilling operations. This integration has been shown to enhance decision-making efficiency by over 50 times \cite{E_Ferrigno_2024}.
            
            % Domain-Specific Knowledge Integration: 
            LLMs empowered by domain-specific knowledge bases (LLM-DSKB) can provide more accurate and relevant insights for industrial applications, addressing the limitations of general LLMs in handling technical issues specific to the oil and gas sector \cite{Huan_Wang_2023}.
            
            
            % Broader Perspective on LLMs in the Industry
        \subsection{Broader Perspective on LLMs in the Industry}
        
            While LLMs offer significant potential to optimize well construction and maintenance processes, their implementation is not without challenges. The integration of LLMs requires careful consideration of data quality and infrastructure to ensure reliable and accurate outputs. Additionally, the lack of domain-specific expertise in existing LLMs can limit their effectiveness in technical applications, necessitating the development of specialized knowledge bases to enhance their utility . Furthermore, the adoption of LLMs in the oil and gas industry must be balanced with considerations of cost, scalability, and the need for ongoing collaboration between industry experts and AI developers to fully realize their potential .
        
        
        % \subsection{aaaaa}
        % \subsection{aaaaa}
        % \subsection{aaaaa}

    \section{Motivação 2 [scispace]}

        [[[[ SCISPACE ]]]]
        PESQUISAR SOBRE DESAFIOS E MOTIVAÇÕES PARA USO DE AGENTES E MULTI-AGENTES (EM DOMÍNIOS ESPECIFICOS?)

    \section{Contribuições 2 [final]}
    
        ****SERÁ FEITO POR ÚLTIMO****
        
    \section{Objetivo do trabalho 2 [final]}

        ****SERÁ FEITO POR ÚLTIMO****

    \section{Organização do trabalho 2 [final]}

        ****SERÁ FEITO POR ÚLTIMO****


    
\chapter{Literature Review} 

    This chapter provides a comprehensive literature review of the key technologies and concepts that form the foundation of this dissertation. It begins with an overview of the applications of Artificial Intelligence (AI) in the Exploration and Production (E\&P) industry. The focus then narrows to LLMs, discussing their architecture and impact. Subsequently, the chapter delves into the RAG technique, which enhances LLMs with external knowledge. It also explores the use of single and multi-agent setups. Finally, the chapter concludes by examining the LLM-as-a-Judge paradigm for evaluating the performance of generative models.

    \section{AI in the Exploration and Production (E\&P) industry}

        The use of AI in the Exploration and Production (E\&P) industry has been extensive. 
        In the last decades the majority of AI applications in the industry involved data mining and neural networks \citep{Bravo2014}. 
        An example is the work by \citep{Gudala2021} on optimization of the properties of the heavy oil flow, through the use of neural networks to optimize flow-influencing parameters.
        Another development was a deep learning workflow proposed by \citep{Gohari2024}, with the generation of synthetic graphic well logs through the application of transfer learning. 
        These developments illustrate the potential of AI to improve processes and the accuracy and efficiency of data analysis \citep{Rahmani2021}.

        Recent studies highlight domain-specific advances in textual AI for geosciences, particularly in Named Entity Recognition (NER) under low-resource conditions. \citet{maze2024textual} proposed a two-phase pipeline that (i) builds a high-quality, semi-automatically labeled dataset via ontology-driven rules, taxonomies, and expert validation, and (ii) augments it using LLM-based rephrasing constrained to preserve entities, cosine-similarity filtering to ensure semantic fidelity and diversity, and entity substitution from curated whitelists. The augmented corpus substantially improved downstream BERT-based NER performance on petroleum technical documents, evidencing the practicality of LLM-driven augmentation for metadata extraction at scale.
    
        Natural Language Processing (NLP) stands at the intersection of computer science and linguistics, representing a domain within artificial intelligence aimed at enabling computers to understand and process human language in a way that is both meaningful and effective \citep{Liddy2001}. 
        This field integrates a diverse range of computational techniques to analyze and represent text at various levels of linguistic detail, striving to emulate human-like language understanding. 
        As an active area of research, traditionally NLP employs multiple layers of language analysis, each contributing uniquely to the interpretation and generation of language, which finds practical applications in various sectors \citep{Liddy2001}.      
        In the O\&G industry, the management of unstructured data, such as texts, images, and documents, is crucial, with NLP and Machine Learning playing key roles.
        Research by \citet{Antoniak2016} and \citet{Castineira2018} has explored the use of NLP to analyze risks and drilling reports.           
        
        Complementing these efforts, \citet{gharieb_role_2024} outline a roadmap for personalized, on-premises LLMs tailored to petroleum engineering and education. Their pipeline benchmarks embeddings and chunking strategies for retrieval. Results indicate that smaller, locally hosted LLMs can deliver competitive summarization and knowledge-integration performance with reduced latency and lower operating costs.
        Extending to drilling operations, \citet{yi2024applications} demonstrate a GPT-based system with retrieval over a curated corpus spanning sensor logs, reports, after-action reviews, and external well construction planning and real-time Q\&A. Reported outcomes include significant time savings in retrieving past incident context (e.g., stuck pipe) and benchmarking parameters (e.g., lateral-section ROP).

    \section{Natural Language Processing} \label{sec:nlp-review}

        NLP is a broad field that covers various tasks to enable computers to process and understand human language \citep{jurafsky2025}. These tasks, which represent specific problems or applications, have been the focus of research for decades, predating the recent surge in LLMs. They range from fundamental challenges like part-of-speech tagging to complex applications like machine translation. This section explores two tasks particularly relevant to this dissertation: Q\&A and Text-to-SQL, both of which have been significantly advanced by recent developments in the field.

        \subsection{Q\&A tasks}    

            Q\&A can be viewed from two complementary perspectives. From the organizational view, Q\&A serves as a mechanism to facilitate knowledge transfer between individuals \citep{Iske2005}. Platforms such as Stack Overflow illustrate how structured question-and-answer workflows support technical communities \citep{Treude2011}. This understanding helps organizations design processes that enhance knowledge transfer and learning.

            From the artificial intelligence perspective, \textit{automated question answering} is a long-standing research area in NLP that aims to answer user queries automatically from available evidence (documents, databases, or parametric model knowledge). 
            % Typical task taxonomies include: (i) \textbf{closed-domain} vs. \textbf{open-domain} QA, depending on whether the knowledge source is restricted; (ii) \textbf{extractive} QA, which selects spans from a context, versus \textbf{abstractive/generative} QA, which produces novel text; and (iii) \textbf{closed-book} QA, where the model relies solely on parametric memory, versus \textbf{open-book} QA that consults external sources.
            % Modern open-domain QA systems commonly follow a \textit{retriever–reader} (or retriever–generator) architecture: a retriever selects candidate passages, and a reader/generator produces the answer. Dense retrieval methods such as DPR \citep{Karpukhin2020} outperform traditional sparse retrieval for passage selection, while Retrieval-Augmented Generation (RAG) \citep{Lewis2020} integrates the retrieved evidence directly into the prompt for an LLM to generate grounded answers. These approaches reduce hallucination and improve factuality compared to purely parametric generation.
            In specialized settings, domain-specific Q\&A adds constraints such as terminology, safety, and privacy. 
            Recent work explores cost-efficient, domain-specific Q\&A with LLMs by optimizing retrieval and context selection \citep{Arefeen2024}. 
            In the petroleum context specifically, applications have leveraged GPT-style models to answer natural-language questions over proprietary corpora and operational documents \citep{Eckroth2023}, aligning with the retrieval-and-generation paradigm adopted in this dissertation. 
            Together, these advancements motivate the use of RAG pipelines for auditable Q\&A in E\&P environments.

        \subsection{Text-to-SQL tasks} 

            Text-to-SQL tasks in the context of artificial intelligence involve the automatic translation of natural language questions or commands into structured SQL (Structured Query Language) queries \citep{Qin2022}. This is an important area in NLP, allowing users to interact with databases using plain language rather than needing to know how to write complex SQL queries.         
                
            The arrival of advanced language models like GPT-3 and GPT-4 \citep{OpenAImodels} has marked a significant leap in Text-to-SQL applications \citep{Singh2023}, demonstrating remarkable capabilities in handling these tasks. This can be attributed to their extensive training on diverse datasets \citep{Deng2021}, which include not only large amounts of text but also structured data like tables and code, enabling the model to understand the intricate relationships between language and data structures. The study by \citep{Deng2023} introduces a pre-training framework for text to SQL translation, emphasizing the alignment between text and tables in Text-to-SQL tasks.



    \section{Intelligent Agents}         

        According to \citet{Russell2020}, an agent is something that performs actions. When it comes to computerized agents (in our case, AI-based), these agents are expected to do more: operate autonomously, perceive the environment, persist over time, adapt to changes, create, and strive to achieve goals. The agent program implements the agent function.

        \citet{Russell2020} present a taxonomy of agent programs that we adopt here in \textit{increasing order of complexity}:
        \begin{enumerate}[label=(\alph*)]
            \item \textbf{Simple reflex}: act based solely on the current percept using condition–action rules.
            \item \textbf{Model-based reflex}: maintain an internal state (a world model) to handle partial observability.
            \item \textbf{Goal-based}: choose actions that achieve explicitly represented goals, enabling lookahead and planning.
            \item \textbf{Utility-based}: select actions to maximize an expected utility over outcomes when trade-offs exist.
            \item \textbf{Learning/adaptive}: improve performance over time by learning components such as perception, model, or utility.
        \end{enumerate}
        The appropriate design depends on the environment and task constraints. In this work, a \textbf{goal-based} agent design was implemented to act toward achieving defined objectives.


        \subsection{Multi-Agent Systems}

            A Multi-Agent System (MAS) extends the concept of a single agent to a collection of agents that interact within a shared environment \citep{Gokulan2010}. A MAS is defined as a loosely coupled network of autonomous problem-solving entities that collaborate to find solutions to problems that are beyond the individual capabilities or knowledge of any single entity \citep{FloresMendez1999}. 

            The structure of a MAS can vary, with different organizational paradigms such as hierarchical structures or coalitions being employed depending on the application \citep{Gokulan2010}. A practical example of a MAS architecture is demonstrated in power system restoration, where a system can be composed of multiple "bus agents" and a single "facilitator agent" \citep{Nagata2002}. In this setup, each bus agent works to restore its local area by negotiating with neighboring agents based on locally available information, while the facilitator agent manages the overall decision-making process, showcasing how a collection of agents with simple, local strategies can cooperate to achieve a complex, global goal \citep{Nagata2002}.



    \section{Large Language Models}         

        LLMs are advanced neural network-based models designed to understand and generate human-like text. 
        They leverage the Transformer architecture introduced in the seminal paper \enquote{Attention is All You Need} by \citet{Vaswani2017}. 
        This architecture relies on self-attention mechanisms, allowing the model to weigh the importance of different words in a sentence effectively. 

        In practice, contemporary generative LLMs are typically \textit{decoder-only} Transformer models, stacking decoder blocks with causal self-attention to autoregressively produce tokens. By contrast, widely used classifiers such as BERT adopt an \textit{encoder-only} configuration that produces contextualized representations for discrimination tasks rather than generation \citep{Devlin2018}. 

        The emergence of LLMs has made it possible to understand and produce textual information. 
        These systems are expected to revolutionize various industries by supporting complex decision-making processes. GPT models \citep{OpenAI2023}, in particular, take advantage of its vast training data to provide human-like responses \citep{Mosser2024}, which can be highly beneficial in contexts requiring natural language understanding and generation. The exponential growth in the size and capability of LLMs in recent years has been remarkable. Models like OpenAI's GPT series have shown significant advancements, moving from millions to hundreds of billions of parameters, which gives them increasingly sophisticated natural language understanding and generation. This advancement is illustrated in Figure~\ref{fig:llm_evolution}. For new models (released after jan/2025), including OpenAI's o3 series and GPT-4.5, Anthropic's Claude 3.7 and 4, and Google's Gemini 2.5 Pro, the exact parameter counts have not been publicly disclosed. 

        \begin{figure}[ht]
            \centering
            \includegraphics[width=0.8\textwidth]{images/llm_evolution.png}
            \caption{The evolution of LLMs.}
            \label{fig:llm_evolution}
        \end{figure}
                
        However, the trajectory of LLM development in 2025 has signaled a shift in focus. While previous advancements were often marked by an exponential increase in parameter counts, the latest generation of models emphasizes sophisticated reasoning capabilities over sheer size. 
        This move away from parameter size as the primary metric of progress underscores a new trend: enhancing the models' ability to perform complex, multi-step reasoning. 
        This is evident in features like the private chain-of-thought mechanisms in OpenAI's models and the "extended thinking" mode in Anthropic's Claude series, indicating that language models are advancing through more intricate cognitive architectures rather than just scaled-up data processing.

        As highlighted by \citet{Singh2023}, the integration of LLM-based solutions, such as conversational chatbots, offers an approach to optimizing operations across various business segments, including drilling, completion, and production.
        \citet{Singh2023} uses LLMs models to extract, analyze, and interpret datasets, enabling generation of insights and recommendations. 

        Despite its widespread impact, language models are not without its limitations. 
        In many industry-specific applications, the critical information required is often proprietary, not shared with third parties, and thus absent from the training data of these LLMs \citep{Mosser2024}. 
        This gap means that GPT models might not have access to the most up-to-date or sensitive information needed for certain tasks. 
        Moreover, due to their probabilistic nature, LLMs can experience hallucinations, producing confident yet incorrect or nonsensical responses based on user input \citep{OpenAI2023}. 
    
    
        \subsection{LLM applications}

            The proliferation of LLMs has led to a diverse array of applications that leverage their ability to understand, generate, and process human language.

            The expansion of the LLM application ecosystem is evident in the significant market growth projections. For instance, one report projects the global LLM market to grow from \$5.62 billion in 2024 to \$35.43 billion by 2030, with a compound annual growth rate (CAGR) of 36.9\% \citep{GrandViewResearch2025}. This rapid expansion is indicative of the immense value and potential that organizations across industries see in these technologies. The applications themselves are becoming increasingly sophisticated, evolving from simple text generation to complex, multimodal systems capable of processing and integrating text, images, and other data formats \citep{Kaddour2023}.
            
            The spectrum of LLM-based applications is broad and continually expanding. Early applications focused on tasks such as text summarization, translation, and sentiment analysis. However, the current generation of LLMs powers a much wider range of tools. These can be broadly categorized into several key areas. Conversational AI, in the form of advanced chatbots and virtual assistants, represents a significant segment of the market, enhancing customer service and user engagement \citep{GrandViewResearch2025}. Content creation is another major application area, where LLMs are employed to generate a variety of materials, from marketing copy and social media posts to technical documentation and even creative writing \citep{V7Labs2025}.            
            
            Furthermore, LLMs are being integrated into more specialized and high-stakes domains. In the legal field, they assist with tasks like contract analysis and legal research. The financial sector utilizes them for fraud detection and market analysis \citep{V7Labs2025}. In software development, LLM-powered tools for code generation and debugging are becoming increasingly prevalent, accelerating development cycles and improving programmer productivity. A key innovation driving the utility of these applications is the advent of techniques like RAG, which allows LLMs to retrieve and incorporate information from external knowledge bases, thereby improving the accuracy and relevance of their outputs \citep{KeywordsAI2025}. The ongoing development of multimodal LLMs is further pushing the boundaries of what is possible, enabling applications that can understand and reason about the world in a more holistic manner \citep{Kaddour2023}.
        
        \subsection{RAG} 

            RAG technique combines LLMs with information retrieval to generate accurate and up-to-date responses, as introduced by \citet{Lewis2020}. 
            It employs a search in a database to find relevant information, overcoming the inherent limitations of LLMs that rely solely on the prior knowledge embedded in the language model during the training phase. 
            With the ongoing evolution of information retrieval, which has evolved from term-based methods to more semantic approaches leveraging deep learning and large datasets to tackle more complex challenges.
            
            A RAG consists of two main components: a retriever and a generator, as illustrated in Figure~\ref{fig:rag_diagram}. The retriever is responsible for finding relevant information from a knowledge base, and the generator uses that information to create a human-like response. 
            
            \begin{figure}[h!]
                \centering
                \includegraphics[width=0.4\textwidth]{images/rag_diagram_vertical.png}
                \caption{A diagram illustrating the RAG process.}
                \label{fig:rag_diagram}
            \end{figure}         

            As elucidated by \citet{Lewis2020}, RAG unites the strengths of pre-trained parametric and non-parametric memory, using a dense vector index and a semantic retriever. 
            As demonstrated by \citet{Li2022} in their analysis, RAG is surpassing traditional generative models in terms of performance across a variety of tasks. The study provides a detailed survey on this topic, emphasizing the fundamental concepts and its applicability in specific contexts.

            New tools have been developed to facilitate the implementation of RAG solutions. \citet{Liu2023} present a toolkit that integrates augmented retrieval techniques into LLMs, including modules for query rewriting, document retrieval, passage extraction, response generation, and fact-checking, enabling the creation of more factual and specific responses. The recent study by \citet{Zhao2023} extends this horizon by examining the incorporation of multimodal knowledge into generative models, exploring the integration of diverse external sources such as images, code, tables, graphs, and audio, to enhance the grounding context and improve usability. It also explores potential future trajectories in this emerging field, marking a relevant contribution to the evolving narrative of RAG and its applications.

            
        \subsection{Multi-Agent Setup} 

            As demonstrated by \citet{xi2023rise}, the pursuit of Artificial General Intelligence\footnote{AGI is the ability of a machine to perform any intellectual task that a human can perform.} (AGI) has significantly benefited from the development of LLM-based agents, capable of sensing, decision-making, and acting across diverse scenarios.  
            His study outline a foundational framework for such agents, consisting of brain, perception, and action components, which can be customized for various applications including single-agent scenarios, multi-agent systems, and human-agent collaboration. 
            The comprehensive survey underscores the crucial role of LLMs in moving towards AGI, suggesting a promising horizon for operational efficiency and decision-making processes in complex organizational settings \citep{xi2023rise}.

            \citet{Li2024} demonstrated that, through a sampling and voting method, the performance of LLMs scales with the number of instantiated agents.
            Another open-source framework is AutoGen \citep{Wu2023}, that enables the creation of LLM multi-agent applications, allowing for customization across various modes. It supports diverse applications in fields such as mathematics, coding, and operations research, demonstrating its effectiveness through empirical studies \citep{Wu2023}.

            
        \section{Evaluation} \label{sec:evaluation-review}

            \subsection{Truthfulness}

                In the evaluation of RAG systems, ensuring the truthfulness of the generated output is a primary concern. \citet{Lin2022} introduces a framework for this purpose. The authors define a truthful answer as one that aligns with literal truth about the real world. This is particularly relevant for RAG systems, which can retrieve and incorporate information from vast and varied sources. An answer is considered truthful if it does not assert any false statements, and informative if it provides relevant information that addresses the user's query.
                
                In \citet{Li2023}, the authors conducted an evaluation to determine the effectiveness of their proposed prompts on the performance of various LLMs. The evaluation employed both automated standard experiments and human studies to assess the impact of emotional stimuli on task performance, truthfulness, and responsibility.

                In the first experiment of this study, human experts assessed each Q\&A pair based on the definitions:

                \begin{quoting}[font={small,itshape},indentfirst=false]
                    \begin{itemize}
                    \item \textbf{Truthfulness}: a metric to gauge the extent of divergence from factual accuracy, otherwise referred to as hallucination \citep{Lin2021}.
                        \subitem 1=“The response promulgates incorrect information, detrimentally influencing the ultimate interpretation”
                        \subitem 2=“A segment of the response deviates from factual accuracy; however,this deviation does not materially affect the ultimate interpretation”
                        \subitem 3=“The response predominantly adheres to factual accuracy, with potential for minor discrepancies that do not substantially influence the final interpretation”
                        \subitem 4=“The response is largely in consonance with factual evidence, albeit with insignificant deviations that remain inconsequential to the final interpretation”
                        \subitem 5=“The response is in meticulous alignment with the facts, exhibiting no deviations”
                                
                    \item \textbf{Performance}: encompasses the overall quality of responses, considering linguistic coherence, logical reasoning, diversity, and the presence of corroborative evidence.
                        \subitem 1 = “The response fails to address the question adequately”
                        \subitem 2 =“The response addresses the question; however, its linguistic articulation is sub-optimal, and the logical structure is ambiguous”
                        \subitem 3 = “The response sufficiently addresses the question, demonstrating clear logical coherence”
                        \subitem 4 = “Beyond merely addressing the question, the response exhibits superior linguistic clarity and robust logical reasoning”
                        \subitem 5 = “The response adeptly addresses the question, characterized by proficient linguistic expression, lucid logic, and bolstered by illustrative examples”\citep{Lin2021}.         
                    \end{itemize}
                \end{quoting}

            \subsection{Precision, Recall, and F1-Score} \label{sec:precision_recall_f1_review}
            Precision, recall, and F1-score are fundamental metrics for evaluating classification tasks, particularly in scenarios with imbalanced datasets. These metrics provide a more nuanced understanding of a model's performance than accuracy alone.

            In a binary confusion matrix, we denote: \textbf{TP} (True Positives), \textbf{FP} (False Positives), \textbf{TN} (True Negatives), and \textbf{FN} (False Negatives). The formulas below use these standard abbreviations.

                \textbf{Precision} measures the accuracy of positive predictions. It is the ratio of correctly predicted positive observations to the total predicted positive observations. A high precision relates to a low false positive rate.
                \begin{equation}
                    \text{Precision} = \frac{\text{TP}}{\text{TP} + \text{FP}}
                    \label{eq:precision}
                \end{equation}

                \textbf{Recall} (or Sensitivity) measures the ability of the model to find all the relevant cases within a dataset. It is the ratio of correctly predicted positive observations to all observations in the actual class.
                \begin{equation}
                    \text{Recall} = \frac{\text{TP}}{\text{TP} + \text{FN}}
                    \label{eq:recall}
                \end{equation}

                The \textbf{F1-score} is the harmonic mean of Precision and Recall. Therefore, this score takes both false positives and false negatives into account. It is a good way to show that a model has a good performance on both metrics.
                \begin{equation}
                    \text{F1-score} = 2 \times \frac{\text{Precision} \times \text{Recall}}{\text{Precision} + \text{Recall}}
                    \label{eq:f1-score}
                \end{equation}


        \subsection{LLM-as-a-Judge}

            The LLM-as-a-Judge paradigm represents a significant shift in the evaluation of NLP systems in general, using a language model as a scalable proxy for human evaluators \citep{li2024llmsasjudgescomprehensivesurveyllmbased}. 
            This approach was developed to overcome the semantic shallowness of traditional metrics like BLEU or ROUGE and the logistical challenges of extensive human annotation \citep{Zheng2023}. 
            By providing a "judge" LLM with a clear rubric and context, it can perform assessments of qualities like coherence, relevance, and factual accuracy \citep{li2024llmsasjudgescomprehensivesurveyllmbased}. 
            This method has proven effective for complex, open-ended tasks where simple string matching is insufficient, with models like GPT-4 demonstrating over 80\% agreement with human preferences in benchmarking studies \citep{Zheng2023}.

            For evaluating RAG systems, the LLM-as-a-Judge framework can be adapted to produce structured, quantitative assessments. 
            In this application, the judge LLM is tasked with comparing the RAG-generated answer against a ground-truth dataset.
            By using a crafted prompt that defines the classification criteria, the judge can systematically categorize each output into classes such as True Positive (TP) (factually consistent with the ground truth), False Positive (FP) (introduces unsupported information), True Negative (TN) (a correct refusal to answer), or False Negative (FN) (missing relevant information). This approach moves beyond subjective scoring towards a more objective evaluation. The prompt used in this work is presented in the code in Appendix~\ref{code:llm-judge}.

            The advantage of this methodology is its ability to translate qualitative judgments directly into a confusion matrix, allowing the calculation of standard metrics such as precision (Equation~\ref{eq:precision}), recall (Equation~\ref{eq:recall}), and F1-score (Equation~\ref{eq:f1-score}). This process establishes a replicable pipeline for benchmarking the factual accuracy of a RAG system at scale. While it is important to acknowledge the potential for inherent biases in LLM judges \citep{Gu2025}, studies show high correlation with human-expert evaluations \citep{li2024llmsasjudgescomprehensivesurveyllmbased}, making it a useful tool for iterative development and system comparison.


% 
\chapter{Revisão 2 [CANCELADO]} 


    \section{Large Language Models}


    \section{aaaaa}


    \section{aaaaa}


    \section{aaaaa}




\chapter{First Experimental Evaluation Cycle}

    \xexeo{Todo capítulo deve ter uma introdução explanatória. "This chapter describes"}
    \vitor{Feito.}

    This chapter describes the first experimental cycle of this research, structured according to the DSR methodology, as detailed in Section~\ref{sec:dsr-application}. It situates the experiment within the established \textbf{context} of knowledge management in the well construction and maintenance domain of a major oil company. The core \textbf{problem} this experiment addresses is the need for an effective mechanism to query complex technical and operational information within large volumes of unstructured data. As a solution, this experiment proposes and implements two distinct \textbf{artifacts}: a single-agent system and a multi-agent architecture, both designed to leverage LLMs for responding to specialized user queries. Finally, the chapter details the \textbf{evaluation} of these artifacts, outlining the methodology, the dataset creation process, and the metrics (Truthfulness, Performance, and LLM Cost) used to assess their capabilities and limitations.

    \todo[inline]{Acho que pode até ser Primeiro Ciclo, ou pode ter um nome como ``Efetividada das LLMs na Solução..''  vai ter que quebrar esse capítulo, que tem muita informação nos conceitos da DSR: no mínimo nos quatro principais: Contexto, Problema, Artefato, Avaliação. Grande parte do contexto e problema já devem ser descritos no capitulo que eu pedi para criar e aqui só faz referência.}

    \todo[inline]{Ok, você foi direto para o experimento mas não disse o que ia fazer. Aqui exatamente cabe o quadro da DSR que eu te mandei: qual o contexto, qual o problema, qual a suposição (de utilidade ou de mundança de contexto), qual os quadro teóricos, qual o(s) artefato(s) proposto(s) e como serão avaliados, vai fechar muito bem.}
    

    \section{Experimental Methodology}

        This section describes the approach and tools employed to investigate the effectiveness of a language model-based agent in responding to specific queries within the domain of well construction and maintenance.
        \xexeo{isso aqui é a validação, mas qual é o problema, qual a proposta, são essas informações que faltam para ficar bem organizado} 
        Firstly, the preparation, selection, and utilization of the data sources are described, explaining how each contributes to the knowledge base from which the agent derives its responses.
        
        \todo{inline}{Aqui seria bom fazer um BPMN do passo a passo do seu experimento, veja a figura 4.1 de\url{https://www.cos.ufrj.br/uploadfile/publicacao/3172.pdf}}

        \subsection{Data Preparation.} 

            This experiment was carried out in the well construction department of a major oil company
            \xexeo{Olha o contexto aqui}.
            The choice of tasks focused on technical knowledge management and data analysis
            \xexeo{Contexto e problema?}. 
            Examples of queries used in the experiment are listed in Table~\ref{table:question_examples}. 

            \xexeo{Exemplos? Não são todas? Quantas foram? Por que foram escolhidas? Tem que explicar como foram criadas essas queries, do jeito mais honesto possível, mesmo que seja pela experiência do autor, por pergunta a colega ou consulta uma coleção de documentos com perguntas e respostas. Tem que listar todas. Se foram MUITAS mesmo, coloca em apêndice.}
            \xexeo{Isso aqui é a exemplificação do problema, aliás, dava até para falar um pouco da criação de datasets de teste lá na revisão da literatura}
            
            The data sources for executing such tasks were chosen to cover a range of operational scenarios in the activity of well construction and maintenance: Operational Knowledge Items, Operational NPTs (Non-Productive Time), and a Collaborator Finder. 

\xexeo{para claridade, eu  separei as linhas da tabela 3.1 com linhas físicas, mas mudei a forma que você estava formatando.}

            % [TENTAR INCLUIR EXEMPLOS DE PERGUNTAS]
            \begin{table}[h]
                \centering
                \sloppy
                \begin{tabular}{|p{.2\linewidth}|p{.8\linewidth}|}
                    \hline
                    \textbf{Task category} & \textbf{Query example} \\   \hline
                    \multirow{9}{*}{Q\&A} & How does the presence of silica in the composition of cement 
                    paste affect its thermal stability at high temperatures? \\ \cline{2-2}
                    & What are the main challenges and risks associated with through tubing plug and abandonment in highly deviated wells? \\ \cline{2-2}
                    & What can cause hydrate formation in the Tree Running Tool  connector during the HCR (High Collapse Resistance) hose flush  before connecting to the Wet Christmas Tree? \\ \cline{2-2}
                    & What can cause the Down Hole Safety Valve to remain open  due to hydrate formation in the control lines? \\ \cline{2-2}
                    & What can cause damage to thread protectors and sealing  areas of pin ends of pipes stored at the coating yard? \\ \cline{2-2}
                    & What can cause high drag and torque off-bottom during  the drilling of a well with high deviation? \\ \cline{2-2}
                    & What precautions should be taken when performing a top check  of the abandonment plug in wells with higher inclination? \\ \cline{2-2}
                    & What are the critical factors to consider when choosing a base  fluid for manufacturing a viscous support plug? \\ \cline{2-2}
                    & What are the best practices for managing drilling parameters  during cement cutting to avoid premature bit wear? \\ 
                    \hline                
                    \multirow{2}{*}{Text-to-SQL} & What was the longest-lasting NPT on rig number 05? \\ \cline{2-2}
                    & How many NPTs occurred on rig number 06 during August 2023? \\ 
                    \hline
                \end{tabular}
                \fussy
                \caption[Query examples used in first cycle.]{Query examples used in first cycle. }
                \label{table:question_examples}
            \end{table}




            \emph{Operational Knowledge Items}: during drilling, completion, and workover interventions, documents called Knowledge Items are written by specialists, as depicted in Fig~\ref{fig:report_example}, which can be of 4 types: Technical Alert, Learned Lesson, Good Practice, and Well Observation. 
            This is a tool for knowledge management, considering the large number and variety of specialists involved and well operations performed.\xexeo{\textbf{Essas explicações sobre o que são os documentos podem vir no capítulo que explica a área e configurar o contexto do DSR}}
            
            \emph{Operational NPTs (Non-Productive Time)}: the second data source refers to data on anomalies that occurred during well interventions, containing information such as the title, description of the event, the well where it occurred, type of operation, responsible sector, involved rig, lost time in hours, and start and end dates of the event. 
            These data are critical for the industry, as NPTs represent periods when the operation of drilling, completion or maintenance is interrupted due to some technical or logistical problem. The identification and analysis of these events are essential for continuous process improvement, cost reduction, and increased operational efficiency. By understanding the causes and circumstances of these incidents, organizations can develop strategies to prevent them in the future, optimizing operation time.\xexeo{capitulo da area}
            
            \emph{Collaborator Finder}: the third data source used in the experiment is a collaborator finder, an important tool within an organization for consulting and managing employee data. 
            This system allows for quick search and identification of employees through information such as name, workplace, company, registration, and role. The importance of this tool for the experiment lies in the possibility of cross-referencing employee data with other information sources for a more complete response by the agent.\xexeo{capitulo da area}
            
            Each of these sources provides inputs for the agent to offer a more accurate and updated view of operations and organizational structure.
            A set of documents and records was randomly selected from each database, over which questions were formulated. For each document, up to 3 questions were generated, resulting in a dataset of tasks of the Q\&A and Text-to-SQL types. Some examples are described in Table \ref{table:question_examples}.\xexeo{revisar para fazer sentido as coisas virem de outro capítulo}

        
            In this work, a goal-based agent\xexeo{Se certificar que estão descritos na revisão da literatura, caso contrário descrever lá} \citep{Russell2020} was implemented with the goal of accurately responding to various queries. 
            The agent operates within an environment equipped with multiple tools for task-specific operations, as shown in Figure~\ref{fig:agent_environment}, and interfaces with users to receive queries.
            
            \begin{figure}[h]
                \centering
                \includegraphics[width=0.75\textwidth]{images/agent_environment_4.png}
                \caption{Schematic of the LLM-based agent interacting with an environment containing tools for task-specific operations, and the Human Agent interface for user interaction and feedback.}
                \label{fig:agent_environment}
            \end{figure}           
            
            Initially, a configuration of agents was implemented as described in Figure~\ref{fig:agent_config_1} using AutoGen Framework \citep{Wu2023} with an architecture that allows information retrieval and user interaction. This system consists of\xexeo{of ... melhor dizer two alguma coisa?}:

            \begin{figure}[h]
                \centering
                \includegraphics[width=.5\textwidth]{images/agent_config_1.png}
                \caption{Chat setup with one User Proxy \citep{Wu2023} and one Assistant.}
                \label{fig:agent_config_1}
            \end{figure}
            

            \begin{itemize}        
                        
                \item \textbf{User Proxy:} represents the interface with the user and with tools to access external databases. The modular nature of the tools allows the User Proxy to be customized and expanded based on the variety of data sources and the specific requirements of the application domain.

                \item \textbf{Agent:} powered by LLMs such as GPT-4 and GPT-3\xexeo{Aqui tem que dizer que é configurável, só ficou claro para mim mais tarde na leitura}, is the analytical engine of the system. This agent interprets the queries received from the User Proxy and formulates responses.
                                    
            \end{itemize}

            
            For each question in the data set, the agent's decision-making process is executed as described in Figure~\ref{fig:diagrama_agente_1}, initially selecting the appropriate tool to respond to a query and, finally, compiling the retrieved information to provide a final answer.
            
            \begin{figure}[h]
                \centering
                \includegraphics[width=0.75\textwidth]{images/agent_diagram_1.png}
                \caption{Decision process of the agent.}
                \label{fig:diagrama_agente_1}
            \end{figure}                        
                    
            In this experiment, three tools were considered in the decision-making process:

            \begin{itemize}
            
                
                \item \textbf{Tool 1 - Knowledge Items Search:} a tool to search for learned lessons that may be relevant to the query. 
                \label{Tool1}
        
                \item \label{Tool2} \textbf{Tool 2 - Employee Search:} functionality that allows the search for information related to collaborators of an organization.
        
                \item \label{Tool3} \textbf{Tool 3 - NPT SQL Query:} Interface for executing SQL queries on a database of operational NPTs.    
                
            \end{itemize}

            In parallel, there is a pathway that allows the LLM Agent to provide a direct response, without the need to resort to other tools, presumably used when the agent already possesses the necessary information. 
            Finally, the agent presents the final response to the user, which is the product of a language model processing, taking as input the user's query and relevant information retrieved and included in the context of the prompt.

             Table~\ref{tab:tabela_inputs} provides a detailed analysis of the performance and accuracy of different models, specifically GPT-3.5-turbo and GPT-4, when queried about the impact of silica on the thermal stability of cement paste at high temperatures.\xexeo{Por que essa pergunta? Isso é um exemplo? Como o multi agente entrou aqui se ainda não falou de multi agente? Melhor deixar para mais tarde ou mostrar só para single-agent?} 
             The table compares both single-agent and multi-agent setups, evaluating the final outputs based on truthfulness, performance, and specialist comments. 
             For each query, the table highlights the relevance and accuracy of the information provided by the models, including any extraneous or unrelated sections noted by the specialists. 
             This \xexeorem[comprehensive]{OLHA UMA PALAVRA INÚTIL TÍPICA DE LLM} comparison allows for an \xexeorem[in-depth]{Outra} assessment of the models' capabilities in generating accurate and relevant responses to technical questions.

            \begin{landscape}
                \begin{table}[h]
                    \small
                    \centering
                    \caption{Query sample with inputs, outputs, and evaluations.}
                    \label{tab:tabela_inputs}
                    \begin{tabular}{| m{3cm} | m{1.5cm} | m{1.5cm} |>{\scriptsize}m{9cm} | m{1.3cm} | m{1.2cm} |>{\scriptsize}m{4cm} |}                    
                    \toprule
                    % \textbf{Input query} & \textbf{Model} & \textbf{Setup} & \textbf{Final output} & \textbf{Truthfulness} & \textbf{Performance} & \textbf{Specialist \ Comments} \\
                    Input query & Model & Setup & Final output & Truth. \newline fulness & Perfor- \newline mance & Specialist \ Comments \\
                    \midrule
                    How does the presence of silica in the composition of cement paste affect its thermal stability at high temperatures? & GPT-3.5-turbo & Single-Agent & The presence of silica in the composition of cement paste can affect its thermal stability at high temperatures due to some characteristics of silica. Silica is an inorganic material that has a stable crystalline structure at high temperatures. When exposed to high temperatures, the silica present in the cement paste can undergo structural transformations, resulting in changes in the properties of the paste.

                    One of the main changes that can occur is the thermal expansion of the cement paste due to the formation of unstable silica compounds. This thermal expansion can lead to cracks and fissures in the structure, compromising the material's integrity and stability.

                    Moreover, the presence of silica in the composition of cement paste can affect the material's thermal resistance. Silica has low thermal conductivity, which can lead to a reduced ability of the cement to withstand high temperatures. This can result in a decrease in the material's strength and, consequently, a reduction in its thermal stability.

                    In summary, the presence of silica (...) & 4 & 5 & Created a section that is irrelevant to the question: "Moreover, the presence of silica in the composition of cement paste can affect the material's thermal resistance. (...)" \\
                    \cline{3-7}
                    &  & Multi-Agent & The presence of silica in the composition of cement paste has a positive effect on thermal stability at high temperatures. Silica is added to cement to mitigate the retrogression of cement strength when exposed to temperatures above 230°F (110°C). Deep wells or those with a high geothermal gradient may require the use of silica cement to ensure the thermal stability of the paste. The use of silica cement is particularly important in surface casing cementing, where the temperature can exceed the critical limit. \ (...) & 2 & 2 & Retrieved an unrelated Lesson Learned on the subject (about plugs). \\ 
                    \cline{2-7}
                    & GPT-4 & Single-Agent & The presence of silica in the composition of cement paste increases thermal stability, especially at high temperatures. Silica is used to transform the cement into a refractory material, capable of withstanding extreme conditions. 
                    
                    The cement (...).

                    However, (...). & 3 & 4 & Although it is true, the item on the side is not relevant to the question: "However, it is important to note that (...)." \\
                    \cline{3-7}
                    &  & Multi-Agent & Silica is added to cement at a ratio of 35\% by weight of cement to mitigate the retrogression (loss of compressive strength) of the cement in the well when subjected to temperatures above 230ºF (110ºC). \ (...) & 5 & 4 & Cited an unrelated (but true) Lesson Learned about plugs, Lesson Learned LA7005. \\
                    \hline
                    \end{tabular}
                \end{table}
            \end{landscape}
            
            

        \subsection{Multi-Agent Architecture.}             
            A second architecture that employs multiple agents was implemented, each having a distinct tool to engage with external data-sources, as illustrated in Figure \ref{fig:agent_config_2}. 
            This architecture also starts with the user’s input. However, as depicted in Figure \ref{fig:diagrama_agente_MultiAgente_2},  the subsequent 'Speaker selection' process determines the specialized agent evaluated as most suitable for answering the user's question.
            
            
            When a query falls within the direct LLM knowledge, the pathway 'No tool needed' is selected, and the corresponding agent responds without further tool engagement.
            The selected agent then 'Compiles and evaluates' the gathered information in the context of the user's query, ensuring a response that is both accurate and context-aware. The final step, 'Provide final answer', is where the multi-agent system converges to deliver the final, coherent answer to the user.
                
            \begin{figure}[h]
                \centering
                \includegraphics[width=.75\textwidth]{images/agent_config_2.png}
                \caption{Chat setup with one Chat Manager and a group of LLM agents.}
                \label{fig:agent_config_2}
            \end{figure}
            
            
            \begin{figure}[h]
                \centering
                \includegraphics[width=1\textwidth]{images/agent_diagram_2.png}
                \caption{Multi-agent decision process.}
                \label{fig:diagrama_agente_MultiAgente_2}
            \end{figure}

            
        \subsection{Evaluation.}             

            For the evaluation process, in line with the evaluation conducted by \citep{Li2023}, a group of 3 specialist engineers analyzed 33 questions\xexeo{Coloca todas na tabela! E faz uma seção de criação de perguntas, ou subseção} and their corresponding answers for each configuration. 
            The experts assessed each Q\&A pair based on the following predefined metrics:

            \begin{itemize}
                
                \item \textbf{Truthfulness:} a metric to gauge the extent of divergence from factual accuracy.\xexeo{tem que descrever em mais detalhes e de onde tirou}
        
                \item  \textbf{Performance:} encompasses the overall quality of responses, considering linguistic coherence, logical reasoning, diversity, and the presence of corroborative evidence.\xexeo{tem que descrever em mais detalhes e de onde tirou}
                
            \end{itemize}
            
            The final grade was determined by averaging the scores of the all entries for each configuration. This evaluation ensured a comprehensive assessment of the models' capabilities.





\section{Proposed Artefacts}
\todo[inline]{Nessa seção descreve as soluções que você criou (pega do texto que já tem}
\subsection{Single-Agent Solution}
\subsection{Multi-Agent Solution}

            
    \section{Results}

    \todo[inline]{Em vez de Results, evaluation?, e você descreve como os artefatos foram avaliados, o texto que tem está bom, é só reorganizar}

\subsection{Data Set Creation}

\todo[inline]{Aqui descreve todo o seu processo de criação das perguntas e todas elas, que tipo são, faz o máximo que puder. Isso é bem legal.}

\subsection{Metrics}

\todo[inline]{Tem que descrever na revisão da literatura de forma geral, mas se forem diferentes por experimento, tem que explicitar aqui as que vai usar.}


        This section provides an analysis of the data collected and answers the research questions. The results are presented in \autoref{tab:tabela_resultados} and are organized according to the objectives of the study, with each objective being addressed in detail.

        The third metric, LLM Cost\xexeo{Isso aqui é uma pergunta de pesquisa tem que entrar de alguma maneira na definição do DSR, lembrando que as avaliações do DSR podem ser mais de uma}, represents\xexeo{Não é represents, já que é o custo mesmo, acho que  corresponds to, ou mesmo só is } the financial cost associated with using OpenAI's API for the language models in each configuration. This metric is measured in US dollars and reflects the computational resources required for each task.\xexeo{Tem que falar alguma coisa que não é o único custo, e quais são os outros e porque esse é importante, isso pode estar descrito no modelo DSR, antes}\xexeo{Esse parágrafo tipicamente aparece na revisão}
                
            
        \begin{table}[h]
            \small % Reduce the font size
            \centering % Center the table on the page
            \caption{Results on Q\&A and Text-to-SQL tasks, including standard deviation (Std). The best metrics are highlighted with \textbf{\underline{bold and underline}}. The second best are highlighted with \textbf{bold}.}
            \label{tab:tabela_resultados}
            \begin{tabular}{|>{\raggedright\arraybackslash}p{2.2cm}|>{\centering\arraybackslash}p{0.9cm}|>{\centering\arraybackslash}p{0.8cm}|>{\centering\arraybackslash}p{0.8cm}|>{\centering\arraybackslash}p{0.8cm}|>{\centering\arraybackslash}p{0.8cm}|>{\centering\arraybackslash}p{0.8cm}|>{\centering\arraybackslash}p{0.9cm}|>{\centering\arraybackslash}p{0.8cm}|>{\centering\arraybackslash}p{0.8cm}|>{\centering\arraybackslash}p{0.8cm}|}
                \hline
                \rowcolor{gray!20}
                \textbf{Task}           & \multicolumn{5}{c|}{\textbf{Single-Agent}}           & \multicolumn{5}{c|}{\textbf{Multi-Agent}} \\ % Merging cells and adding heading
                \textbf{Model}          & \textbf{LLM Cost} & \textbf{Truth.} & \textbf{Std} & \textbf{Perf.} & \textbf{Std} & \textbf{LLM Cost} & \textbf{Truth.} & \textbf{Std} & \textbf{Perf.} & \textbf{Std} \\ \hline
                \cellcolor{gray!20} Q\&A & & & & & & & & & &\\
                GPT-3.5-turbo            & 0.005             & 2.94              & 1.48 & 3.94          & 1.09 & 0.02              & 4.09              & 1.22 & 3.82 & 0.98 \\
                GPT-4                   & 0.12              & \textbf{3.88}     & 1.41 & \textbf{4.06} & 1.30 & 0.45              & \underline{\textbf{4.57}} & 0.79 & \underline{\textbf{4.43}} & 0.79 \\
                \cellcolor{gray!20} Text-to-SQL & & & & & & & & & &\\
                GPT-3.5-turbo            & 0.009             & 4.13              & 1.41 & 4.44          & 1.03 & 0.02              & \textbf{4.29}     & 1.20 & \textbf{4.29} & 1.33 \\
                GPT-4                   & 0.10 & \underline{\textbf{4.56}} & 0.96 & \underline{\textbf{4.63}} & 0.81 & 0.51      & 3.20              & 1.99 & 3.70 & 1.89 \\ \hline
            \end{tabular}
        \end{table}

        % A análise comparativa entre as arquiteturas de agentes únicos e multiagentes para o RAG, utilizando modelos GPT-3.5-turbo e GPT-4, revelou insights em relação as métricas de truthfulness, performance e custos do modelo de linguagem. 
        The comparative analysis between single and multi-agent setups for RAG, using GPT-3.5-turbo and GPT-4 models, revealed insights regarding the metrics of truthfulness, performance, and costs of the language model.

        
        \subsection{Truthfulness} 
            % Na avaliação da métrica truthfulness, observa-se diferenças significativas entre as configurações de agente único e multi-agente em ambas as tarefas de Q\&A e Text-to-SQL. Os resultados são ilustrados nas Figuras ~\ref{fig:truthfulness_QA} e ~\ref{fig:truthfulness_text2sql}.
            In assessing the truthfulness metric, significant differences are noted between the single and multi-agent settings in both Q\&A and Text-to-SQL tasks. The results are illustrated in Figures \ref{fig:truthfulness_QA} and \ref{fig:truthfulness_text2sql}.
            For Q\&A tasks, GPT-4 in a multi-agent configuration significantly exceeded the performance of the single-agent with a truthfulness score of 4.57 compared to 3.88. The GPT-3.5-turbo model showed distinct results between the two configurations, with the multi-agent surpassing the single-agent with scores of 4.09 and 2.94, respectively.
            In terms of Text-to-SQL queries, a different outcome was observed. GPT-4 single-agent achieved a score of 4.56, while the same model in the multi-agent configuration obtained 3.20, highlighting a limitation for the multi-agent in this task. Conversely, the GPT-3.5-turbo maintained a more balanced performance between configurations, scoring 4.29 for multi-agent and 4.13 for single-agent.
            
            
            
            \begin{figure}[h]
                \centering
                \begin{minipage}{.49\textwidth}
                    \centering                
                    % \framebox{
                    \includegraphics[width=1\linewidth]{images/truthfulness_QA.png}
                    % }
                    \caption{Truthfulness and standard deviation in Q\&A tasks by LLM model and agent configuration. \\ }
                    % \caption{Truthfulness and standard deviation in Q\&A tasks by LLM model and agent configuration.}
                    \label{fig:truthfulness_QA}
                \end{minipage}%
                \hspace{0.2cm}
                \begin{minipage}{.49\textwidth}
                    \centering
                    % \framebox{
                    \includegraphics[width=1\linewidth]{images/truthfulness_text2sql.png}
                    % }
                    \caption{Truthfulness and standard deviation in Text-to-SQL tasks by LLM model and agent configuration.}
                    \label{fig:truthfulness_text2sql}
                \end{minipage}
            \end{figure}

            
            
            
        \subsection{Performance}        
        
            The evaluation of LLM performance \citep{Li2023} in the tasks of Q\&A and Text-to-SQL reveals trends which are similar to the truthfulness results. 
            As shown in Figures \ref{fig:performance_QA} and \ref{fig:performance_text2sql} and summarized in \ref{tab:tabela_resultados}, the text performance in single and multi-agent setups was compared using the GPT-3.5-turbo and GPT-4 models.        
            For Q\&A tasks, the multi-agent setup shows a performance boost compared to the single-agent setup. In particular, the multi-agent GPT-4 achieves a performance score of 4.43, which is higher than the single-agent GPT-4 score of 4.06. This pattern is consistent with the GPT-3.5-turbo, where the multi-agent system also surpasses the single-agent system, scoring 3.82 and 3.94, respectively. These findings emphasize the effectiveness of the multi-agent approach in handling technical user queries.
                    
            \begin{figure}[h]
                \centering
                \begin{minipage}{.48\textwidth}
                    \centering                
                    % \framebox{
                        \includegraphics[width=1\linewidth]{images/performance_QA.png}
                    % }
                    \caption{Performance and standard deviation in Q\&A tasks by LLM model and agent configuration.\\}
                    % \caption{Performance and standard deviation in Q\&A tasks by LLM model and agent configuration.}
                    \label{fig:performance_QA}
                \end{minipage}
                \hspace{0.2cm}
                \begin{minipage}{.48\textwidth}
                    \centering
                    % \framebox{
                    \includegraphics[width=1\linewidth]{images/performance_text2sql.png}
                    % }
                    \caption{Performance and standard deviation in Text-to-SQL tasks by LLM model and agent configuration.}
                    \label{fig:performance_text2sql}
                \end{minipage}%
            \end{figure}

        % \subsection{Tempo de Resposta}
        
            % O tempo médio de resposta dos agentes foi registrado, mostrando uma resposta rápida para a maioria das perguntas. No entanto, algumas consultas mais complexas exigiram tempos de processamento mais longos.
            
        % \subsection{LLM Cost}
        
        \subsection{LLM Cost} 
            Language model services are typically composed by a values per token. For instance, GPT-4 model costs US\$30.00 (input) and US\$60.00 (output) per 1 million tokens received and sent, respectively.        
            The single-agent architecture demonstrated substantially lower costs for both Q\&A and Text-to-SQL tasks compared to the multi-agent setup as shown in Figure~\ref{fig:truthfulness_vs_cost_vs_config_model}. For instance, the average cost of the GPT-4 model \citep{OpenAI2023} for a Q\&A task was \$0.12 per processed question for the single-agent, while the multi-agent recorded an average cost of \$0.45. This trend of higher costs for the multi-agent architecture was also maintained for Text-to-SQL tasks, with an average cost of \$0.51 for the multi-agent architecture in contrast to \$0.10 for the single agent.
            The higher token count and cost for multi-agent setting is due to the inclusion of intermediate calls, for example, when the "Agent Selector" needs to decide which agent to pass the turn to. All the message history is passed to the LLM at this stage, substantially increasing the number of tokens submitted and response time.
        

            \begin{figure}[h]
                \centering              
                % \framebox{
                    \includegraphics[width=0.75\textwidth]{images/truthfulness_vs_cost_vs_config_model.png}
                % }
                \caption{Average LLM costs and Truthfulness per completed task according to setup and model.}
                \label{fig:truthfulness_vs_cost_vs_config_model}
            \end{figure}
            
            

    \section{Discussion}

        The comparison between single and multi-agent systems revealed significant differences in terms of performance and cost:
        
        \subsection{General Performance.}     
            The results indicate that for Q\&A tasks in the context of O\&G, truthfulness measure was 28\% higher with the multi-agent architecture compared to single. 
            However, for Text-to-SQL tasks, this trend was inverted, where the single-agent scored 15\% higher.

            These findings suggest that for Q\&A tasks, the multi-agent setup may be more advantageous in terms of providing truthful information, particularly when utilizing the more advanced GPT-4 model. 
            Conversely, in Text-to-SQL tasks, the GPT-4 model in a single-agent configuration proved more effective. 
            This might imply that the added complexity of managing multiple agents in some tasks does not necessarily lead to improved performance in responses, underscoring the importance of carefully selecting the agent configuration based on the task type and specific features of the language model used.
                
        \subsection{Cost-Performance Analysis.}
            While the multi-agent system shows higher truthfulness in Q\&A tasks, it is crucial to consider the associated costs. 
            % For instance, a truthfulness score of 3.8 in a multi-agent setup might be acceptable if the cost is significantly lower than achieving a 4.5 score with a single-agent system. 
            % Based on our analysis, we recommend using a multi-agent system for Q\&A tasks when the budget allows for it and truthfulness is a critical factor. For Text-to-SQL tasks, a single-agent system is preferable due to its higher performance and potentially lower cost. Decision-makers should consider setting a cost-performance threshold to guide the choice of system configuration, ensuring that the benefits justify the expenses involved.
            To provide a clearer comparison, let us consider the score/cost ratios. For Q\&A tasks using GPT-4, the single-agent configuration yields a ratio of 32.33 truthfulness points per dollar, compared to 10.16 for the multi-agent setup. This indicates that while the multi-agent system shows a 17.8\% improvement in truthfulness, it comes at a 275\% increase in cost.
            
            Based on our analysis, we recommend using a multi-agent system for Q\&A tasks when the budget allows for it and accuracy is a critical factor. 
            However, decision-makers should consider setting a cost-performance threshold to guide the choice of system configuration, ensuring that the benefits justify the expenses involved.\xexeo{TEm que deduzir a necessidade de fazer um experimento antes levando essas coisas em consideração}


        \subsection{Model Performance Variations.}
            Interestingly, our results show that GPT-3.5-turbo outperforms GPT-4 in certain tasks, particularly in the Text-to-SQL multi-agent configuration, despite GPT-4's larger size and more extensive training. 
            This unexpected performance could be attributed to several factors. 
            First, GPT-3.5-turbo may have undergone more specific fine-tuning for structured query tasks, allowing it to excel in Text-to-SQL scenarios. 
            Additionally, GPT-3.5-turbo's training data might be more recent or more relevant to the specific domain of our study. 
            Another possibility is that the smaller model size of GPT-3.5-turbo allows for faster processing and more efficient handling of the multi-agent setup, resulting in better performance in some contexts.

            However, it is important to note that GPT-4, when used in a multi-agent setup, demonstrated more consistent truthfulness and performance, as evidenced by its reduced standard deviation in results. 
            This consistency can be particularly advantageous in applications where reliability and accuracy are critical. 
            Multi-agent systems have the advantage of maintaining separate contexts for different aspects of a task. \xexeo{Você pode suportar essa afirmação com uma citação?}
            This compartmentalization can lead to better handling of complex, multi-faceted queries, as each agent can focus on its specific context without being overwhelmed by irrelevant information. However, this advantage may be offset in tasks like Text-to-SQL, where maintaining a unified context of the database schema and query structure is crucial, possibly explaining the better performance of single-agent setups in this task.
            Furthermore, the multi-agent architecture inherently involves multiple stages of information processing, which can serve as natural filtering mechanisms.
            As information passes from one agent to another, irrelevant or low-quality data may be naturally filtered out, leading to more refined and accurate final outputs. 
            This could explain the superior performance in filtering irrelevant information observed in multi-agent setups.
        
        
        \subsection{Economic Efficiency.} 
            % The multi-agent architecture demonstrated a cost, on average, 3.7 times higher than that of the single-agent. This increase is due to the addition of intermediate calls to the language model, for instance, in the Speaker Selection phase, where a request is necessary to select the next agent for dialogue. Additionally, more than one iteration between agents for action planning and tool use can occur. From the model standpoint, the average cost of completed tasks (from user question to final answer) using GPT-4 was 21 times higher than those performed with GPT-3.5-turbo, an expected difference due to the varying costs of each service.
            The multi-agent architecture incurs significantly higher costs compared to the single-agent system, primarily due to additional intermediate calls to the language model and multiple iterations between agents for action planning. 
            Also, the cost differences between using GPT-4 and GPT-3.5-turbo are substantial, with GPT-4 being notably more expensive\xexeo{Dizer x vezes mais caro em julho de 2025}.
            As detailed in the Cost-Performance Analysis section, the truthfulness per dollar ratio highlights the economic trade-offs between single-agent and multi-agent systems. While the multi-agent system offers improvements in truthfulness, this comes at a considerable increase in cost, impacting the overall economic efficiency.
            
            % To provide a clearer picture of economic efficiency for a large company with 40.000 knowledge workers, using GPT-4 could result in an annual cost of approximately \$5M, while GPT-3.5-turbo would cost around \$500.000. These estimates assume an average usage pattern of 10.000 tokens per worker per day and highlight the significant impact of model choice on operational costs.
            % To provide a clearer picture of economic efficiency, we can examine the truthfulness per dollar ratio. The single-agent system yields 32.33 truthfulness points per dollar, whereas the multi-agent system yields 10.16 truthfulness points per dollar. This comparison highlights that, although the multi-agent system shows a 17.8\% improvement in truthfulness, it comes at a 275\% increase in cost, resulting in a lower truthfulness per dollar ratio compared to the single-agent system.

            % For a large company with 40,000 knowledge workers, using GPT-4 could result in an annual cost of approximately \$5M, while GPT-3.5-turbo would cost around \$500,000. These estimates assume an average usage pattern of 10,000 tokens per worker per day and highlight the significant impact of model choice on operational costs.
            For a large company with 40,000 knowledge workers, the choice of model and architecture significantly impacts annual costs. 
            Using GPT-4 in a single-agent configuration could result in an annual cost of approximately \$4.38 million, while GPT-3.5 would cost around \$438,000. However, when employing a multi-agent architecture, the costs increase substantially. \xexeo{Pulou como você chegou a esses números, eu estou achando até baixos, qual a conta que você fez (empregados x consultas x etc...)}
            The multi-agent configuration with GPT-4 would escalate the annual cost to \$16.425 million, representing a dramatic increase due to the 3.75 times higher token usage. 
            Similarly, GPT-3.5 in a multi-agent setup would cost \$1.642 million. 
            These estimates assume an average usage pattern of 10,000 tokens per worker per day and underscore the significant financial implications of adopting a multi-agent system, which, while potentially offering performance benefits, comes with a considerable increase in LLM costs.\xexeo{Ah! Tá no fim do parágrafo e precisa ser no início. Além disso tem que dizer quantos dias de trabalho por ano, e o custo de cada token em uma data}

            In summary, while multi-agent systems and more advanced models like GPT-4 offer improvements in performance, the economic efficiency, as measured by truthfulness per dollar, may favor single-agent systems and less costly models like GPT-3.5-turbo, depending on the specific application and budget constraints.\xexeo{In summary é o parágrafo típico das LLMs... Mas é isso mesmo. Porém tem que colocar um ponto: o custo dos modelos está caindo barbaramente com o aparecimento de novos modelos no topo de desempenho e novas tecnologias tem permitido alcançar resultados de ótima qualidade com máquinas muito menores, o que também derruba o custo. Pode até citar o exemplo do DeepSeek (buscando na literatura o desempenho x custo dele}
            
        % \subsection{Challenges and Limitations}
        
        \subsection{Challenges and Limitations.}     
            During the evaluation of the agents, several challenges and limitations were identified.

            \setlength{\parindent}{1em}
            \textbf{\textit{Contextualization and Interpretation.}} 
                In many cases, the single-agent solution had difficulty understanding the context of the question. For example, a question about cementing was interpreted in the context of the construction industry, a theme to which the language models were more exposed during the training phase. 
                However, the multi-agent structure, with its well-defined roles, better understood the questions and showed superior performance in Q\&A tasks, corroborating the findings of \citep{Li2024}.
            
            \setlength{\parindent}{1em}
            \textbf{\textit{Filtering Irrelevant Information.}} 
                The agent often receives irrelevant documents along with important ones in the prompt context, and it is up to the LLM to ignore these. 
                For example, when asked about alternatives to accelerate the curing time of cement paste without compromising its integrity at high temperatures, the RAG system retrieved a document that included information about batch cementing to ensure homogeneity during manufacturing and pumping. 
                While this information is true, it was not relevant to the specific question asked. 
                In this aspect, the multi-agent solution performed better at discarding such irrelevant information, focusing more accurately on the task at hand. 
                Other possible solutions include improving the accuracy of semantic search by adjusting a minimum threshold for similarity measures or through re-ranking techniques such as those proposed by \citep{Carraro2024} and \citep{Sun2023}.
            
            \setlength{\parindent}{1em}
            \textbf{\textit{Hallucination.}} 
                During the evaluation of our system, we encountered instances where the agent produced hallucinated information instead of utilizing the appropriate tool to retrieve accurate data, as in \citep{Bilbao2023}. 
                For example, when asked, "How many anomalies occurred on rig number 05 during August 2023?" the agent was expected to use the Text-to-SQL tool to query the database. 
                However, it bypassed this tool and generated a fabricated response, stating that 5 anomalies occurred, along with detailed descriptions of fictional events. The correct answer, as retrieved from the database, was that 7 anomalies occurred. This hallucination likely resulted from the agent's reliance on its internal knowledge rather than external data retrieval. 

                In terms of hallucination statistics, our analysis revealed that for Q\&A tasks, hallucinations occurred in 9.6\% of cases and 3.8\% for partially hallucinated. 
                In contrast, Text-to-SQL tasks exhibited a lower hallucination rate, with only 3.6\% of responses containing hallucinated information and 96.4\% being accurate. 
                These findings highlight the variation of susceptibility to hallucination in different types of tasks, highlighting the need for targeted strategies to mitigate this problem.
            
            \setlength{\parindent}{1em}
            \textbf{\textit{Industry Jargon:}}
                Specifically analyzing the activity of drilling and completion of offshore wells, the main challenge is the inherently complex and technical nature of the data involved. 
                There were instances of incorrect interpretation of information, likely due to the use of terms, expressions, and themes specific to well construction, to which the language model had little or no exposure during training phase. 
                A possible solution is the implementation of specialized models, which has been pointed out in gray literature as a trend for the coming years \citep{Shah2024, Meena2023, Ghosh2023}.
            
            \setlength{\parindent}{1em}
            \textbf{\textit{Tools vs. Performance:}} 
                It was identified during the experiments that agents with a high amount of tools showed a decline in overall performance. 
                This can be attributed to the added context to the prompts. 
                As the context length increases, the model's ability to accurately interpret and respond diminishes.
                This is a limitation of current language models, where longer contexts can lead to a dilution of relevant information and increased difficulty in maintaining coherence and accuracy. 
                This conclusion is currently qualitative, as these metrics were not addressed in this experiment.

            
            \setlength{\parindent}{1em}
            \textbf{\textit{Queries Involving Proper Names:}}
                In queries involving people's names, it was not possible to retrieve relevant documents using semantic search. 
                For example, when asked to identify the employee associated with a specific key and list knowledge items they registered in the system, the RAG system incorrectly attributed knowledge items to the wrong author\xexeo{O RAG ou a LLM usando o RAG, não ficou claro}. 
                This highlights the difficulty in accurately retrieving information based on proper names, which can be complicated by variations in accentuation, abbreviation, and formatting.\xexeo{tem evidências disso em outros artigos?}
                A potential solution to be explored is the use of Self-Query Retriever \citep{LangchainSelfQuery2023}, implementing a hybrid search with metadata filters (including proper names) and semantic retrieval of the rest of the query. 
                It is also suggested, in these cases, to use the \citep{Levenshtein1966} distance to handle possible variations in the spelling of names. 
                This approach could improve the accuracy of retrieving documents related to specific individuals, ensuring that the correct information is associated with the right person.
                
        
        % \subsection{Practical Implications}
        \subsection{Practical Implications.} 
                    The findings from our study have significant practical implications for the O\&G sector, and potentially for other industries characterized by complex and technical data environments:
                    
                \begin{itemize}
                
                    \item \textbf{Enhanced Decision-Making Support:}
                        Our results indicate that multi-agent systems provide a 28\% higher truthfulness measure in Q\&A tasks. This can be particularly beneficial for decision-making in well engineering, where accurate and truthful information is critical.
                        Implementing multi-agent systems in decision-making processes can lead to more reliable and informed decisions, thereby reducing the risk of errors and enhancing operational safety and efficiency.
                    
                    \item \textbf{Balancing Performance and Economic Efficiency:}
                        While multi-agent systems offer superior performance in terms of truthfulness, they come with a cost that is 3.7 times higher on average compared to single-agent systems.
                        This highlights the importance of a strategic approach in selecting agent configurations based on specific tasks and budget constraints. 
                        % For instance, single-agent systems might be more cost-effective for Text-to-SQL tasks where they have shown to perform 15\% better. 
                        A detailed cost-benefit analysis reveals that for Q\&A tasks using GPT-4, the single-agent configuration yields a ratio of 32.33 truthfulness points per dollar, compared to 10.16 for the multi-agent setup. While the multi-agent system shows a 17.8\% improvement in truthfulness, this comes at a 275\% increase in cost. The efficiency varies significantly by task type; in Text-to-SQL tasks, the GPT-4 single-agent outperforms the multi-agent by 42.5\% in truthfulness while costing 80.4\% less. 
                        % These quantitative insights emphasize the need for careful consideration of task requirements and budget constraints when choosing between single and multi-agent configurations.
                        
                    \item \textbf{Reflection and Critic Agents:}
                        A promising approach to enhance the performance of these agents is the use of reflection \citep{Shinn2023}, a method where agents verbally reflect on task feedback signals and maintain this reflective text in an episodic memory buffer to improve decision-making in subsequent trials. Critic agents are a way to implement reflection in a multi-agent setup. This type of agent is challenging to apply in Q\&A tasks over private technical data, as commercial LLMs (OpenAI, Google Bard, and others) have not been deeply trained in the domain and struggle to provide relevant and precise critiques, reinforcing the trend toward increased use of domain-specific models \citep{Shah2024, Meena2023, Ghosh2023}.                
                        
                    \item \textbf{Task-Specific Agent Configuration:}
                        The study highlights that the complexity of managing multiple agents does not always lead to better performance. In some cases, a single-agent setup might be more effective.
                        This insight can guide the development and deployment of AI systems, ensuring that the configuration of agents is tailored to the specific requirements of the task, thereby optimizing both performance and cost.           
                        
                    \item \textbf{Potential for Broader Application:}
                        The insights gained from this study are not limited to the O\&G sector but can be applied to other industries with similar technical complexities, such as aerospace, pharmaceuticals, and renewable energy.
                        By adopting multi-agent systems in these industries, organizations can improve decision-making, knowledge management, and operational efficiency, driving innovation and competitiveness.             
                    
                \end{itemize}
                    
                
        \subsection{Future Directions.} 

            This work indicates possible pathways for enhancing RAG architectures in O\&G sector. 
            
            \begin{itemize}
            
                \item \textbf{Enhancement of IR Semantic Techniques:}
                    There is a critical need to develop more sophisticated semantic search technologies. Future efforts should focus on enhancing the precision of information retrieval by filtering out irrelevant content more effectively. This will ensure that agents can provide more accurate and contextually appropriate responses, crucial for technical domains such as O\&G.
                    
                \item \textbf{Development of Domain-Specific Models:}
                    Specialized models tailored specifically to the O\&G and other domains, such as biomedical engineering \citep{Pal2024}, could significantly improve the handling of specific jargon and complex technical data, while reducing LLM costs \citep{Arefeen2024}. Future research should aim to develop and train these models to better understand and interpret the unique language and data types found in O\&G, enhancing the overall accuracy of agent responses.
                    
                \item \textbf{Optimization of Tool Use in Agent Performance:}
                    The relationship between the quantity of tools available to an agent and its performance needs further exploration. Future studies should quantify the impact of tool availability on agent efficacy and efficiency, aiming to optimize tool use without overwhelming the agent or diluting performance quality.
                    
                \item \textbf{Integration of Advanced Name Recognition Techniques:}
                    Queries involving proper names pose a significant challenge in semantic search. Integrating advanced retrieval techniques, such as Self-Query Retrievers \citep{LangchainSelfQuery2023} and \citep{Levenshtein1966} distance algorithms, could improve the handling of these queries. Future research should focus on enhancing name recognition capabilities to ensure that agents can accurately retrieve and utilize correct information, especially in scenarios where precision is paramount.
                    
                \item \textbf{Extension to Other Complex Domains:}
                    The potential applications of multi-agent systems are not limited to the O\&G sector. Future research should explore the adaptation and implementation of these systems in other complex and technical domains, such as aerospace, pharmaceuticals, and renewable energy. Investigating how these systems can support decision-making in these areas will provide valuable insights into their versatility and adaptability.
                    
                \item \textbf{Hybrid Model Experimentation:}
                    Combining the strengths of single and multi-agent systems could yield significant benefits. Future directions should include experimenting with hybrid models that integrate the robustness and depth of multi-agent interactions with the simplicity and efficiency of single-agent systems. This hybrid approach could potentially offer a balanced solution, maximizing performance while managing costs and complexity.
                    

            \end{itemize}
            
            By pursuing these directions, future research can significantly advance the development of multi-agent systems, not only enhancing their application in the O\&G sector but also expanding their utility across various technologically intensive activities.
            

% \chapter{Second Experimental Evaluation Cycle}

% \section{Design Science Research Framework}

% \section{Context and Problem Statement}

%     \subsection{Context}

%     \subsection{Problem}

% \section{Proposed Artifacts}

% \section{Evaluation}

% 	\subsection{Evaluation Methodology}

% 	\subsection{Data Set Creation}

% 	\subsection{Evaluation Metrics}

%     \subsection{Results}

%     \subsection{Discussion}


\chapter{Second Experimental Evaluation Cycle}
\label{chap:second_experiment}

This chapter describes the second experimental cycle of this research, building upon the findings of the first cycle detailed in Chapter 3. The rapid evolution of generative AI frameworks and models, along with the insights gained previously, prompted a more advanced and rigorous evaluation. This second phase employs non-agentic workflows as a baseline, introduces a more quantitative evaluation methodology, and leverages an automated assessment process based on the "LLM-as-judge" concept \citep{Gu2025}.

The use of "LLM-as-a-judge" was driven by the sheer volume of responses requiring evaluation. With four configurations, two models, and three executions for each of the 33 [AJUSTAR] questions, a total of 792 [AJUSTAR] responses were generated. Manually assessing this volume of data would have been impractical. Furthermore, previously used metrics like `truthfulness` had become less critical. This metric was highly relevant when models frequently hallucinated, a problem that is far less prevalent in the current generation of LLMs, shifting the focus to precision and recall of factual information.

[TODO: GERAR BPMN!!]

\section{Design Science Research Framework}

    This second experimental cycle adheres to the Design Science Research (DSR) methodology, focusing on refining the artifacts and evaluation based on the outcomes of the first cycle.

    \begin{description}
        \item[Context] The operational environment of well construction engineering, where practitioners require efficient and reliable access to vast amounts of technical and ESG-related information.

        \item[Problem] The first experimental cycle revealed several limitations, including the subjective nature and scalability issues of expert-based evaluation, the need to compare agentic systems against simpler non-agentic baselines, and the challenge of ensuring consistent performance. This second cycle addresses the problem of developing a more robust, scalable, and objective method for evaluating and comparing different LLM-based architectures for domain-specific Q\&A.

        \item[Proposed Artifacts] Four distinct architectures were designed and implemented to compare different strategies for information retrieval and reasoning:
        \begin{itemize}
            \item A non-agentic \textbf{Linear-Flow} RAG pipeline.
            \item A non-agentic \textbf{Linear-Flow with a Router} to direct queries.
            \item A \textbf{Single-Agent} architecture, refined from the first experiment.
            \item A \textbf{Multi-Agent Supervisor} architecture for distributed reasoning.
        \end{itemize}

        \item[Evaluation] The artifacts are evaluated using an automated pipeline. An "LLM-as-a-judge" assesses the generated answers against a ground-truth dataset. The evaluation is based on quantitative information retrieval metrics: \textbf{Precision}, \textbf{Recall}, and \textbf{F1-Score}.
    \end{description}

\section{Context and Problem Statement}

    \subsection{Context}

    As established in the previous chapters, this research is situated within the oil and gas industry, specifically in the domain of well construction and maintenance. Engineers and specialists in this field must navigate a complex information landscape, drawing from operational reports, ESG alerts, and documented best practices (Learned Lessons) to make critical decisions. The effectiveness of these decisions hinges on the speed and accuracy with which relevant information can be retrieved and synthesized.

    \subsection{Problem}

    The first experimental cycle confirmed the potential of LLM-based agents but also highlighted key challenges. The manual, expert-led evaluation process was time-consuming and difficult to scale. Furthermore, the performance differences between single and multi-agent systems suggested that a more granular analysis was needed, including a comparison with non-agentic RAG workflows to establish a performance baseline. Therefore, the central problem for this second cycle is to design and execute a more rigorous, automated, and scalable evaluation to definitively compare the efficacy of various agentic and non-agentic architectures in this specialized domain.

% \section{Proposed Artifacts}

%     To address the research problem, four distinct artifacts were developed using the LangChain and LangGraph frameworks. These architectures represent a spectrum of complexity, from simple sequential pipelines to collaborative multi-agent systems.

\section{Proposed Artifacts}

    To address the research problem, four distinct artifacts were developed, representing a spectrum of complexity from simple sequential pipelines to collaborative multi-agent systems. 
    
    \subsection{System Architecture Overview}

    The experimental system was implemented using \citet{Langchain2025} and \citet{Langgraph2025} frameworks specialized in language model orchestration. This modular design allows for the systematic and reproducible evaluation of different components and workflows. Key layers of the architecture include:

    \begin{itemize}
        \item \textbf{Experiment Orchestration:} Manages the execution loop, iterating through all combinations of questions, models, and setups.
        \item \textbf{Agent Workflow Frameworks:} Defines the logic for each of the four proposed artifacts using LangGraph to create cyclical graphs for agentic behavior.
        \item \textbf{Tool Integration:} A standardized interface providing agents with access to external knowledge sources. This layer enables consistent semantic search over domain-specific vector stores, ensuring that performance differences are attributable to architectural choices rather than variations in data access.
        \item \textbf{Prompt Engineering:} A library of system messages and prompt templates designed to guide the LLM's reasoning process for each specific task within the workflows.
        \item \textbf{State Management and Logging:} Captures the complete execution trace of each run, including intermediate steps, tool calls, and final outputs. This observability is essential for understanding not just the final output, but the process by which each architecture arrived at its answer.
    \end{itemize}


    % \subsection{Artifact 1: Linear-Flow}

    %     The Linear-Flow architecture represents the simplest RAG design, where user input is processed in a strictly sequential manner, as shown in Figure \ref{fig:diagrama_linear_flow}. In this setup, a single LLM call handles the generation of search queries for all available tools. The prompts for each tool are concatenated, creating a large and complex context for the LLM. While straightforward, this can lead to performance degradation as the context length increases.

    %     \begin{figure}[h]
    %         \centering
    %         \includegraphics[width=0.8\textwidth]{images_exp2/diagrams/diagrama_linear_flow.png}
    %         \caption{Linear-Flow architecture. PTn indicates the prompt for Tool n.}
    %         \label{fig:diagrama_linear_flow}
    %     \end{figure}

    \subsection{Artifact 1: Linear-Flow}

        The \textbf{Linear-Flow} architecture represents the simplest non-agentic RAG design, serving as a performance baseline. As shown in Figure \ref{fig:diagrama_linear_flow}, user input is processed in a strictly sequential manner. The user's query is handled by a single LLM step, which contains all the instructions needed to generate search queries for every available tool.
        
        \begin{figure}[h]
            \centering
            \includegraphics[width=0.8\textwidth]{images_exp2/diagrams/diagrama_linear_flow.png}
            \caption{Linear-Flow architecture. PTn indicates the prompt for Tool n.}
            \label{fig:diagrama_linear_flow}
        \end{figure}

        Because the instruction prompts for all tools are aggregated into a single call, the resulting context for the LLM becomes notably extensive and complex. While this approach is straightforward to implement, its primary drawback is the potential for performance degradation as the context length increases, which can dilute the model's focus and lead to less precise retrieval queries.
        

    % \subsection{Artifact 2: Linear-Flow with Router}
        
    %     This architecture (Figure \ref{fig:diagrama_linear_w_router}) introduces a routing mechanism to improve upon the basic linear flow. A preliminary LLM call acts as a "router," analyzing the user's question and directing it to the most appropriate tool or sequence of tools. This allows for smaller, more specialized prompts for each tool, reducing context length and potentially improving the quality of the generated search queries.

    %     \begin{figure}[h]
    %         \centering
    %         \includegraphics[width=0.8\textwidth]{images_exp2/diagrams/diagrama_linear_w_router.png}
    %         \caption{Linear-Flow with Router architecture.}
    %         \label{fig:diagrama_linear_w_router}
    %     \end{figure}


    \subsection{Artifact 2: Linear-Flow with Router}
    
        The \textbf{Linear-Flow with Router} paradigm (Figure \ref{fig:diagrama_linear_w_router}) extends the basic pipeline by introducing a routing mechanism to create a descentralized, non-agentic workflow. This architecture first directs a user's question to a "router" node, which is a preliminary LLM call tasked with analyzing the query and determining the most appropriate tool or sequence of tools to use.

        \begin{figure}[h]
            \centering
            \includegraphics[width=0.8\textwidth]{images_exp2/diagrams/diagrama_linear_w_router.png}
            \caption{Linear-Flow with Router architecture.}
            \label{fig:diagrama_linear_w_router}
        \end{figure}

        This design enables the distribution of complex instruction prompts into smaller, more specialized nodes. Instead of one large prompt, several targeted sub-queries are generated, each dispatched to its respective tool. This approach offers two main advantages:

        \begin{itemize}
            \item \textbf{Specialization:} Each tool receives a query tailored to its specific function, leading to more accurate and relevant retrieval results.
            \item \textbf{Reduced Context:} By breaking down the master prompt, each LLM call operates on a smaller, more focused context, mitigating performance issues associated with long context windows.
        \end{itemize}


    % \subsection{Artifact 3: Single-Agent}

    %     The Single-Agent architecture (Figure \ref{fig:diagrama_single_agent}) represents a centralized agentic approach. A single LLM agent manages the entire question-answering process, autonomously deciding which tools to use, in what order, and how to synthesize the information. This artifact tests the end-to-end reasoning capabilities of an LLM within a unified context, allowing it to perform multi-step reasoning without the overhead of inter-agent communication.

    %     \begin{figure}[h]
    %         \centering
    %         \includegraphics[width=0.5\textwidth]{images_exp2/diagrams/diagrama_single_agent.png}
    %         \caption{Single-Agent architecture.}
    %         \label{fig:diagrama_single_agent}
    %     \end{figure}
    \subsection{Artifact 3: Single-Agent}

        The \textbf{Single-Agent} architecture (Figure \ref{fig:diagrama_single_agent}) embodies a centralized agentic approach, building on the lessons from the first experimental cycle. In this setup, a single LLM agent manages the entire question-answering process. It has access to the full suite of tools and autonomously makes decisions about which to invoke, in what order, and how to synthesize the retrieved information into a final answer.
        
        \begin{figure}[h]
            \centering
            \includegraphics[width=0.5\textwidth]{images_exp2/diagrams/diagrama_single_agent.png}
            \caption{Single-Agent architecture.}
            \label{fig:diagrama_single_agent}
        \end{figure}    

        The design emphasizes \textbf{end-to-end reasoning within a unified context}, allowing the model to maintain the same "thought process" from start to finish. This artifact tests the capability of a standalone LLM agent to manage a RAG workflow, balancing the tool calling for different knowledge sources, all without the communication overhead required by multi-agent systems.
        
    
    % \subsection{Artifact 4: Multi-Agent Supervisor}
        
    %     This setup (Figure \ref{fig:diagrama_multiagente_supervisor}) implements a collaborative, hierarchical agent system. A "supervisor" agent decomposes the user's query and delegates sub-tasks to a team of specialized agents, each with access to a specific tool (e.g., a "Knowledge Items Agent," "ESG Alert Agent"). The supervisor then integrates the findings from the specialists to generate a final, comprehensive answer. This architecture explores the benefits of distributed cognition and task specialization.

    %     \begin{figure}[h]
    %         \centering
    %         \includegraphics[width=0.6\textwidth]{images_exp2/diagrams/diagrama_multiagente_supervisor.png}
    %         \caption{Multi-Agent Supervisor architecture with four specialist agents.}
    %         \label{fig:diagrama_multiagente_supervisor}
    %     \end{figure}

    \subsection{Artifact 4: Multi-Agent Supervisor}
    
        The \textbf{Multi-Agent Supervisor} setup (Figure \ref{fig:diagrama_multiagente_supervisor}) implements a collaborative, hierarchical system to explore the benefits of distributed cognition. This architecture consists of two main components:        

        \begin{enumerate}
            \item \textbf{A Supervisor Agent:} This master agent receives the user's query, analyzes it, and orchestrates the workflow by delegating these tasks to the appropriate specialist agents.
            \item \textbf{Specialist Agents:} A team of agents, each focusing on a specific domain of knowledge or reasoning skill. For this experiment, each specialist was tied to a single tool (e.g., a "Learned Lessons Agent," an "HSE Alert Agent").
        \end{enumerate}
        
        \begin{figure}[h]
            \centering
            \includegraphics[width=0.6\textwidth]{images_exp2/diagrams/diagrama_multiagente_supervisor.png}
            \caption{Multi-Agent Supervisor architecture with four specialist agents.}
            \label{fig:diagrama_multiagente_supervisor}
        \end{figure}

        The supervisor orchestrates the collaboration, integrates the findings from each specialist, and synthesizes the potentially divergent information into a single, coherent final answer. This framework is designed to mimic real-world expert collaboration and tests whether decomposing a problem and assigning its parts to dedicated specialists yields a more accurate result.
        
        
\section{Evaluation}

    The evaluation phase was designed to be automated, scalable, and objective, addressing the limitations of the first experimental cycle.

    \subsection{Evaluation Methodology}

        The core of the evaluation is an automated execution loop (detailed in Algorithm \ref{alg:execution_loop}) that runs each of the 33 questions through every combination of artifact (4 setups) and model (2 models), repeating each run three times to account for stochasticity.

        \begin{algorithm}[h]
        \caption{Experiment Execution Loop}
        \begin{algorithmic}[1]
        \Require questions, setups, models
        \Ensure results
        \Function{RunExperiment}{}
            \State $results \gets \{\}$
            \ForAll{$question \in questions$}
                \State $ground\_truth \gets question.ground\_truth$
                \ForAll{$setup \in setups$}
                    \ForAll{$model \in models$}
                        \For{$i \in 1 \dots 3$} \Comment{Execute 3 times for consistency}
                            \State $agent \gets \text{InitializeAgent}(setup, model)$
                            \State $response \gets agent.\text{ProcessQuestion}(question)$
                            \State $metrics \gets \text{EvaluateResponse}(response, ground\_truth)$
                            \State Store $metrics$ and $response$ in $results$
                        \EndFor
                    \EndFor
                \EndFor
            \EndFor
            \State \Return $\text{AggregateResults}(results)$
        \EndFunction
        \end{algorithmic}
        \label{alg:execution_loop}
        \end{algorithm}

        The quality of each generated response is assessed using the \textbf{LLM-as-a-Judge} approach. A powerful LLM (GPT-4) is prompted to act as an impartial evaluator, comparing the generated answer against the ground-truth answer. The judge decomposes both texts into atomic statements and classifies them to build a confusion matrix, from which the final metrics are calculated. The full prompt for the LLM-as-a-Judge can be found in Appendix \ref{code:llm-judge}.

    \subsection{Data Set Creation}

        The experiment utilizes a curated dataset developed in collaboration with domain experts.
        \begin{itemize}
            \item \textbf{Questions Dataset:} A set of 17 questions reflecting real-world information needs of well engineers. Each question is paired with a manually created, expert-validated "ground-truth" answer.
            \item \textbf{Knowledge Bases:} The artifacts were given access to three distinct, pre-processed knowledge sources from within the organization, vectorized for semantic search:
            \begin{itemize}
                \item \textbf{Learned Lessons:} A repository of learned lessons, best practices, and operational alerts.
                \item \textbf{HSE Alerts:} A collection of ESG alerts and incident reports.
                \item \textbf{Operational Reports:} A database of detailed daily operational reports from drilling rigs.
            \end{itemize}
        \end{itemize}

    \subsection{Evaluation Metrics}

        To provide a quantitative and objective assessment, the following information retrieval metrics were calculated for each response based on the LLM-as-a-Judge's analysis:
        \begin{itemize}
            \item \textbf{Precision:} Measures the accuracy of the information presented in the generated answer. It is the ratio of correct statements (True Positives) to the total number of statements made. 
            $$ \text{Precision} = \frac{\text{TP}}{\text{TP} + \text{FP}} $$
            \item \textbf{Recall:} Measures the completeness of the answer. It is the ratio of correct statements retrieved to the total number of statements available in the ground truth.
            $$ \text{Recall} = \frac{\text{TP}}{\text{TP} + \text{FN}} $$
            \item \textbf{F1-Score:} The harmonic mean of Precision and Recall, providing a single, balanced measure of overall performance.
            $$ \text{F1-Score} = 2 \times \frac{\text{Precision} \times \text{Recall}}{\text{Precision} + \text{Recall}} $$
        \end{itemize}

    \subsection{Results}

        To ensure a robust evaluation and account for the inherent non-determinism of language models, each of the 17 questions in the dataset was processed three times for every model and configuration combination. This experimental design resulted in a total of 408 executions (17 questions $\times$ 2 models $\times$ 4 configurations $\times$ 3 runs). Each of the 408 generated answers was then compared to a ground truth answer to calculate performance metrics.

        The results presented in this section are derived from this set of runs. For each of the 136 unique combinations of question, model, and configuration, the best-performing run (out of three) was selected based on the F1-Score. The final metrics reported in Table \ref{tab:performance_metrics} represent the average of these best-run scores across all 17 questions for each of the eight model-configuration pairs. This approach presents a clear view of the potential of each setup, with the F1-Score serving as the primary metric for performance evaluation.

        \begin{landscape}
            \begin{table}[H]
            \centering
            \caption{Detailed performance metrics by model and agent configuration. The best result for each metric is highlighted in bold and underlined. For the inferior model, the best result is only underlined.}
            \label{tab:performance_metrics}
            % \resizebox{\textwidth}{!}{%
            \begin{tabular}{@{}llcccccccccccc@{}}
                \toprule
                \multirow{2}{*}{\textbf{Model}} & \multirow{2}{*}{\textbf{Configuration}} & \multicolumn{4}{c}{\textbf{F1-Score}} & \multicolumn{4}{c}{\textbf{Precision}} & \multicolumn{4}{c}{\textbf{Recall}} \\
                \cmidrule(l){3-6} \cmidrule(l){7-10} \cmidrule(l){11-14}
                & & Mean & Std. Dev. & Min & Max & Mean & Std. Dev. & Min & Max & Mean & Std. Dev. & Min & Max \\
                \midrule
                \multirow{4}{*}{GPT-4o} & Linear-Flow (Baseline) & 0.581 & 0.204 & 0.000 & 1.000 & 0.656 & 0.262 & 0.000 & 1.000 & 0.548 & 0.201 & 0.000 & 1.000 \\
                & Linear-Flow w/ Router & \textbf{\underline{0.702}} & 0.202 & 0.333 & 1.000 & \textbf{\underline{0.805}} & 0.185 & 0.400 & 1.000 & \textbf{\underline{0.674}} & 0.242 & 0.286 & 1.000 \\
                & Single-Agent & 0.643 & 0.213 & 0.364 & 1.000 & 0.751 & 0.198 & 0.400 & 1.000 & 0.618 & 0.240 & 0.294 & 1.000 \\
                & Multi-Agent & 0.664 & 0.214 & 0.286 & 1.000 & 0.746 & 0.221 & 0.286 & 1.000 & 0.630 & 0.231 & 0.286 & 1.000 \\
                \midrule
                \multirow{4}{*}{GPT-4o-mini} & Linear-Flow (Baseline) & 0.534 & 0.208 & 0.000 & 0.923 & 0.604 & 0.262 & 0.000 & 1.000 & 0.516 & 0.216 & 0.000 & 0.923 \\
                & Linear-Flow w/ Router & \underline{0.604} & 0.155 & 0.333 & 1.000 & 0.676 & 0.196 & 0.300 & 1.000 & \underline{0.602} & 0.206 & 0.267 & 1.000 \\
                & Single-Agent & 0.576 & 0.184 & 0.308 & 1.000 & 0.719 & 0.214 & 0.286 & 1.000 & 0.544 & 0.227 & 0.231 & 1.000 \\
                & Multi-Agent & 0.596 & 0.182 & 0.348 & 1.000 & \underline{0.687} & 0.198 & 0.400 & 1.000 & 0.578 & 0.201 & 0.235 & 1.000 \\
                \bottomrule
            \end{tabular}%
            % }
            \end{table}
        \end{landscape}


        \begin{figure}[h]
            \centering
            \includegraphics[width=0.7\textwidth]{images_exp2/bar_best_f1_by_model_and_configuration.png}
            \caption{Best F1-Score by model and configuration.}
            \label{fig:best_f1_by_model_and_configuration}
        \end{figure}



    \subsection{Discussion}














        % The results of this second experimental cycle provide clearer insights into the effectiveness of different LLM architectures.
        % \begin{itemize}
        %     \item \textbf{Agentic Architectures are Superior:} Both the Single-Agent and Multi-Agent Supervisor setups significantly outperformed the non-agentic Linear-Flow baselines. This demonstrates that for complex, multi-faceted questions requiring information from diverse sources, a simple "retrieve-then-read" RAG pipeline is insufficient. The iterative reasoning, planning, and tool-use capabilities inherent to agentic frameworks are crucial for achieving high performance.

        %     \item \textbf{Specialization and Collaboration Pay Off:} The Multi-Agent Supervisor architecture achieved the highest F1-score. This suggests that decomposing a complex problem and assigning sub-tasks to specialized agents is a highly effective strategy. The supervisor acts as a reasoning orchestrator, leveraging the focused expertise of each team member, which leads to more comprehensive and accurate answers.

        %     \item \textbf{Model Capability Matters:} Across all architectures, the GPT-4 model consistently outperformed GPT-3.5-turbo. This is particularly evident in the more complex agentic setups, where the superior reasoning and instruction-following capabilities of the more advanced model can be fully leveraged.

        %     \item \textbf{Limitations and Future Work:} While the automated evaluation methodology proved scalable and objective, it has its own limitations. The LLM-as-a-Judge is not infallible and can be influenced by its own internal biases or the phrasing of the evaluation prompt. Furthermore, the ground-truth answers, while expert-validated, represent just one possible correct response. Future work could explore ensemble judging methods to increase evaluation reliability and investigate more advanced agentic structures, such as those that allow for dynamic team formation or incorporate self-correction loops.
        % \end{itemize}

        % In summary, this second, more rigorous experimental cycle confirms that for complex, domain-specific tasks, sophisticated agentic architectures like a Multi-Agent Supervisor provide the best performance. Their ability to decompose problems and orchestrate specialized agents leads to more precise and complete answers than either non-agentic pipelines or monolithic single-agent systems.

        
        % [TODO: FURTHER RESULTS WILL BE ADDED HERE]



\chapter{Conclusões 1}

    Os resultados deste estudo destacam o potencial das arquiteturas multiagente baseadas em LLMs no setor de O\&G, especialmente no domínio da engenharia de poços. A capacidade de processar e responder a consultas complexas abre caminho para uma transformação digital significativa na área.
    
    Nossa análise comparativa de arquiteturas de agente único e multiagente, utilizando GPT-3.5-turbo e GPT-4, revela um panorama detalhado de trade-offs entre desempenho e eficiência econômica. Os sistemas multiagente demonstram uma veracidade 28\% maior em tarefas de perguntas e respostas (Q\&A), especialmente com GPT-4, em comparação com sistemas de agente único. No entanto, eles incorrem em custos de LLM que são, em média, 3,7 vezes maiores devido às complexidades da comunicação entre agentes. Em contraste, os sistemas de agente único se destacam em tarefas de Text-to-SQL, apresentando um desempenho 15\% melhor do que as configurações multiagente. Essa dinâmica de custo-benefício exige uma consideração cuidadosa ao implementar RAG em cenários do mundo real, onde precisão e restrições financeiras devem ser equilibradas.
    
    Destacamos vários desafios encontrados durante nossos experimentos, incluindo questões de contextualização, necessidade de filtragem de informações mais refinada e a persistência de alucinações. Esses desafios sublinham a necessidade de pesquisas contínuas em áreas como modelos especializados em domínios específicos, técnicas avançadas de busca semântica e arquiteturas híbridas que combinem as forças dos sistemas de agente único e multiagente.
    
    As implicações práticas deste estudo vão além do setor de O\&G. Os insights alcançados aqui são aplicáveis a qualquer domínio intensivo em conhecimento que lide com grandes volumes de dados técnicos. Ao focar em aprimorar os mecanismos de recuperação, desenvolver LLMs específicos de domínio e otimizar as interações entre agentes e ferramentas, pavimentamos o caminho para soluções RAG mais eficazes, confiáveis e econômicas em diversos setores.
    
    Os principais pontos do estudo são os seguintes: sistemas multiagente oferecem superior veracidade em tarefas de Q\&A, embora a um custo significativamente maior. Arquiteturas de agente único, por outro lado, se destacam em tarefas de Text-to-SQL. Apesar das vantagens, persistem vários desafios, incluindo questões de contextualização, filtragem, alucinação e vocabulário específico de domínio.
    
    Pesquisas futuras devem focar no desenvolvimento de modelos especializados, no avanço das técnicas de recuperação e na exploração de arquiteturas híbridas. As lições aprendidas deste estudo têm implicações mais amplas e podem se estender a outros domínios técnicos complexos. Ao abordar as limitações identificadas neste estudo e abraçar as tendências emergentes em sistemas multiagente e tecnologia RAG, podemos desbloquear seu potencial total, revolucionando a tomada de decisões, a gestão do conhecimento e a eficiência operacional em indústrias complexas em todo o mundo.
    
    


\chapter{Conclusões 2 AAA}

    ...


\backmatter
\bibliographystyle{en-coppe-unsrt}
\bibliography{bib}

\appendix


% \chapter{Um apêndice}

    % Segundo a norma da ABNT (Associação Brasileira de Normas Técnicas), a definição e utilização de apêndices e anexos seguem critérios específicos para a organização de documentos acadêmicos e técnicos.
    
    % Apêndice: O apêndice é um texto ou documento elaborado pelo autor do trabalho com o objetivo de complementar sua argumentação, sem que seja essencial para a compreensão do conteúdo principal do documento. O uso de apêndices é indicado para incluir dados detalhados como questionários, modelos de formulários utilizados na pesquisa, descrições extensas de métodos ou técnicas, entre outros. Os apêndices são identificados por letras maiúsculas consecutivas, travessão e pelos respectivos títulos. A inclusão de apêndices visa a fornecer informações adicionais que possam ajudar na compreensão do estudo, mas cuja presença no texto principal poderia distrair ou desviar a atenção do leitor dos argumentos principais.

\chapter{Experiment 2}
    

    \section{Code for LLM-as-a-Judge}

\begin{lstlisting}[style=mystyle, language=Python, caption={C\'{o}digo para LLM-as-a-Judge}, label={code:llm-judge}]
class Confusion_Matrix(TypedDict):  # type: ignore
    true_positive: list[str]
    false_positive: list[str]
    true_negative: list[str]
    false_negative: list[str]

def calculate_metrics(llm, question, history):
    if type(question['Ground Truth']) == str:
        prompt_confusion_matrix = f"""                
            Voce recebera os seguintes parametros:
            Pergunta: a pergunta do usuario
            Resposta Ideal: a resposta considerada correta por um humano
            Resposta do sistema: a resposta fornecida pelo sistema baseado em IA

            Pegue a resposta do sistema, separe em afirmacoes e classifique cada afirmacao entre as opcoes abaixo:
            True Positive (TP): as afirmacoes corretas feitas pelo sistema, ou seja, que estao presentes na resposta ideal.
            False Positive (FP): as afirmacoes incorretas ou irrelevantes feitas pelo sistema, ou seja, que nao estao presentes na resposta ideal.

            Pegue a resposta ideal, separe em afirmacoes e classifique cada afirmacao entre as opcoes abaixo:
            True Negative (TN): Nao se aplica, deixar vazio.
            False Negative (FN): as afirmacoes que constam na resposta ideal, mas nao foram feitas pelo sistema.

            Importante:
            - Voce deve gerar listas de afirmacoes para cada categoria. 
            - Voce deve quebrar as respostas do sistema e a ideal em afirmacoes objetivas.
            - Se as respostas do sistema ou a ideal contiverem frases grandes com mtas afirmacoes, analisar cada afirmacao separadamente.
            - Uma afirmacao nao pode estar em mais de uma categoria.
            - Ignore coisas na resposta do sistema que nao sao afirmacoes objetivas, como por exemplo citacoes de fontes e links.

            Vamos la!

            ##################

            Pergunta: {{{{
            {question['Question']}
            }}}}

            Resposta ideal: {{{{
            {question['Ground Truth']}
            }}}}

            Resposta do sistema: {{{{
            {history[-1].content}
            }}}}
        """
        response = llm.with_structured_output(Confusion_Matrix).invoke(prompt_confusion_matrix)

        true_positive_count = len(response.get('true_positive', []))
        false_positive_count = len(response.get('false_positive', []))
        true_negative_count = len(response.get('true_negative', []))
        false_negative_count = len(response.get('false_negative', []))

        try:
            precision = true_positive_count / (true_positive_count + false_positive_count)
        except ZeroDivisionError:
            precision = 0
        
        try:
            recall = true_positive_count / (true_positive_count + false_negative_count)
        except ZeroDivisionError:
            recall = 0

        try:
            f1_score = 2 * (precision * recall) / (precision + recall)
        except ZeroDivisionError:
            f1_score = 0
            
        print("\\n\\nPrecision: "+str(precision))
        print("Recall: "+str(recall))
        print("F1 Score: "+str(f1_score))

        print("\\nAnswer Size Ratio: "+str(question['Answer Size (% of GT)']))

        state.precision = precision
        state.recall = recall
        state.f1_score = f1_score
        state.ground_truth = question['Ground Truth']
        state.statements = response
\end{lstlisting}

    \section{Results}
    \label{sec:exp2_appendix}

        % \subsection{Precision}

            % \subsubsection{Average Precision by Configuration}            % \begin{figure}[H]
            %     \centering
            %     \includegraphics{images_exp2/precision/bar_avg_precision_by_configuration.png}
            %     \caption{Average precision by configuration.}
            %     \label{fig:bar_avg_precision_by_configuration}
            % \end{figure}

            % \subsubsection{Average Precision by Model}
            % \begin{figure}[H]
            %     \centering
            %     \includegraphics{images_exp2/precision/bar_avg_precision_by_model.png}
            %     \caption{Average precision by model.}
            %     \label{fig:bar_avg_precision_by_model}
            % \end{figure}

            \subsubsection{Best Precision by Model and Configuration}
            \begin{figure}[H]
                \centering
                \includegraphics[scale=0.75]{images_exp2/precision/bar_best_precision_by_model_and_configuration.png}
                \caption{Best precision by model and configuration.}
                \label{fig:bar_best_precision_by_model_and_configuration}
            \end{figure}

            \subsubsection{Best Precision by Question Index and Configuration}
            \begin{figure}[H]
                \centering
                \includegraphics[scale=0.75]{images_exp2/precision/best_precision_by_question_index_and_configuration.png}
                \caption{Best precision by question index and configuration.}
                \label{fig:best_precision_by_question_index_and_configuration}
            \end{figure}

            \subsubsection{Best Precision by Question Index and Model}
            \begin{figure}[H]
                \centering
                \includegraphics[scale=0.75]{images_exp2/precision/best_precision_by_question_index_and_model.png}
                \caption{Best precision by question index and model.}
                \label{fig:best_precision_by_question_index_and_model}
            \end{figure}

            \subsubsection{Facet Histogram of Precision by Model}
            \begin{figure}[H]
                \centering
                \includegraphics[width=\textwidth]{images_exp2/precision/facet_hist_precision_by_model.png}
                \caption{Facet histogram of precision by model.}
                \label{fig:facet_hist_precision_by_model}
            \end{figure}

            \subsubsection{Facet Histogram of Precision by Model (best of 3)}
            \begin{figure}[H]
                \centering
                \includegraphics[width=\textwidth]{images_exp2/precision/facet_hist_precision_by_model_best_precision.png}
                \caption{Facet histogram of precision by model (best of 3).}
                \label{fig:facet_hist_precision_by_model_best_precision}
            \end{figure}

            % \subsubsection{Heatmap of Precision by Question and Model}
            % \begin{figure}[H]
            %     \centering
            %     \includegraphics[scale=0.75]{images_exp2/precision/heatmap_precision_by_question_and_model.png}
            %     \caption{Heatmap of precision by question and model.}
            %     \label{fig:heatmap_precision_by_question_and_model}
            % \end{figure}

            \subsubsection{Histogram of All Precisions}
            \begin{figure}[H]
                \centering
                \includegraphics[scale=0.75]{images_exp2/precision/hist_precision_all.png}
                \caption{Histogram of all precisions.}
                \label{fig:hist_precision_all}
            \end{figure}

            \subsubsection{Line Plot of Precision by Question Index and Model}
            \begin{figure}[H]
                \centering
                \includegraphics[width=\textwidth]{images_exp2/precision/line_precision_by_question_index_and_model.png}
                \caption{Line plot of precision by question index and model.}
                \label{fig:line_precision_by_question_index_and_model}
            \end{figure}

            \subsubsection{Boxplot of Precision by Model and Configuration}
            \begin{figure}[H]
                \centering
                \includegraphics[scale=0.75]{images_exp2/precision/precision_boxplot_by_model_and_configuration.png}
                \caption{Boxplot of precision by model and configuration.}
                \label{fig:precision_boxplot_by_model_and_configuration}
            \end{figure}

            \subsubsection{Precision by Model}
            \begin{figure}[H]
                \centering
                \includegraphics[scale=0.75]{images_exp2/precision/precision_by_model.png}
                \caption{Precision by model.}
                \label{fig:precision_by_model}
            \end{figure}

            \subsubsection{Precision by Model and Configuration}
            \begin{figure}[H]
                \centering
                \includegraphics[scale=0.75]{images_exp2/precision/precision_by_model_and_configuration.png}
                \caption{Precision by model and configuration.}
                \label{fig:precision_by_model_and_configuration}
            \end{figure}

            \subsubsection{Line Plot of Precision by Question Index and Configuration}
            \begin{figure}[H]
                \centering
                \includegraphics[width=\textwidth]{images_exp2/precision/precision_lineplot_by_question_index_and_configuration.png}
                \caption{Line plot of precision by question index and configuration.}
                \label{fig:precision_lineplot_by_question_index_and_configuration}
            \end{figure}

            \subsubsection{Scatter Plot of Precision vs. Total Time}
            \begin{figure}[H]
                \centering
                \includegraphics[scale=0.75]{images_exp2/precision/scatter_precision_vs_total_time.png}
                \caption{Scatter plot of precision vs. total time.}
                \label{fig:scatter_precision_vs_total_time}
            \end{figure}

            \subsubsection{Scatter Plot of Precision vs. Total Token Count Input}
            \begin{figure}[H]
                \centering
                \includegraphics[scale=0.75]{images_exp2/precision/scatter_precision_vs_total_token_count_input.png}
                \caption{Scatter plot of precision vs. total token count input.}
                \label{fig:scatter_precision_vs_total_token_count_input}
            \end{figure}

            \subsubsection{Swarm Plot of Precision by Model and Configuration}
            \begin{figure}[H]
                \centering
                \includegraphics[scale=0.75]{images_exp2/precision/swarm_precision_by_model_and_configuration.png}
                \caption{Swarm plot of precision by model and configuration.}
                \label{fig:swarm_precision_by_model_and_configuration}
            \end{figure}

            \subsubsection{Violin Plot of Precision by Model and Configuration}
            \begin{figure}[H]
                \centering
                \includegraphics[scale=0.75]{images_exp2/precision/violin_precision_by_model_and_configuration.png}
                \caption{Violin plot of precision by model and configuration.}
                \label{fig:violin_precision_by_model_and_configuration}
            \end{figure}

        % \subsection{Recall}

        %     TEXT

        %     ...

        %     ... 

            
        %     TEXT

        %     ...

        %     ... 

        

\renewcommand{\appendixname}{Appendix}
\appendix

% 
\chapter{Um Anexo}
    Segundo a norma da ABNT (Associação Brasileira de Normas Técnicas), a definição e utilização de apêndices e anexos seguem critérios específicos para a organização de documentos acadêmicos e técnicos.
    
    Anexo: O anexo, por sua vez, consiste em um texto ou documento não elaborado pelo autor, que serve de fundamentação, comprovação e ilustração. O uso de anexos é apropriado para materiais como cópias de artigos, legislação, documentos históricos, fotografias, mapas, entre outros, que tenham relevância para o entendimento do trabalho do autor. Assim como os apêndices, os anexos são identificados por letras maiúsculas consecutivas, travessão e pelos respectivos títulos. Eles são utilizados para enriquecer o trabalho com informações de suporte, garantindo que o leitor tenha acesso a documentos complementares importantes para a validação dos argumentos apresentados no texto principal.

\end{document}

%% 
%%
%% End of file `example.tex'.
